\chapter{Συνδεσμολογία}

%
% Κινητήρες
%
Σύμφωνα με εγχειρίδιο χρήσης της \textcite{hitec02}, όλοι οι κινητήρες της
λειτουργούν σε τάση 4.8--6V δεχόμενοι τετραγωνικό σήμα ελέγχου με διαμόρφωση
PWM (\textenglish{\emph{Pulse Width Modulation}}) 3--5V σε συχνότητα ανανέωσης
50Hz. Η διάρκεια των παλμών κυμαίνεται μεταξύ 0.9ms και 2.1ms, με παλμούς
των 1.5ms να χρησιμοποιούνται για την ακινητοποίησή τους \parencite{hitec02}.



~\\Galvanic isolation -- maybe not, afterall\\

Η απομόνωση των δύο κυκλωμάτων διατηρώντας κάποια σημεία επαφής για την
ανταλλαγή ορισμένων σημάτων μπορεί να επιτευχθεί με τη χρήση οπτικού συζεύκτη.

~\\More on optocouplers\\

Εφόσον μέσω του οπτικού συζεύκτη διέρχεται το παλμικό σήμα ελέγχου του κινητήρα,
είναι απαραίτητο να διαβεβαιωθεί ότι μπορεί να υποστηρίξει τις επιθυμητές
συχνότητες. ~~~

Όπως έχει αναφερθεί σε προηγούμενη ενότητα, ένα φωτοτρανζίστορ απαιτεί κάποιο
χρόνο για να ανταποκριθεί σε εναλλαγές των προσπίπτουσων ακτίνων (χρόνος ανόδου,
$t_r$, και καθόδου, $t_f$) \nref.

~\\Network module -- WIZ811MJ\\

~\\Manual URLs\\

Αποτελεί μία έτοιμη λύση για την προσθήκη δικτυακών δυνατοτήτων στην υλοποίηση.
W5100 + MAG-JACK

Βασικός λόγος για την επιλογή του component -(έναντι)-

Αναλαμβάνει τη διασύνδεση του W5100 με συνδετήρα RJ-45, ταλαντωτή και διάφορων
άλλων στοιχείων και παρέχει διάφορες συνδέσεις ευκολίας. Ένα παράδειγμα σχετικά
με το τελευταίο, το W5100 διαθέτει έναν
ακροδέκτη για την επιλογή του ολοκληρωμένου ως slave από το master του διαύλου
SPI, το SCS. Ωστόσο, διαθέτει και έναν επιπρόσθετο ακροδέκτη, τον SEN, για τη
γενικότερη ενεργοποίηση των υποσυστημάτων του ολοκληρωμένου για λειτουργία σε
SPI. Το WIZ811MJ παρέχει ακροδέκτη μόνο για το \nbar{SCS} ενώ διαθέτει κατάλληλη
συνδεσμολογία ώστε η αλλαγή της τιμής του να προκαλεί το αναμενόμενο σήμα στο
SEN, το οποίο είναι active high.

Το WIZ811MJ διαθέτει 40 ακροδέκτες από τους, συνολικά, 80 του W5100 ενώ,
παράλληλα, αποτελεί μία έτοιμη για χρήση λειτουργική μονάδα.

Το σχήμα \nref{} παρουσιάζει τους ακροδέκτες της μονάδας.

Header J1
Ακροδέκτες ιδιαίτερου ενδιαφέροντος είναι οι 1, 2 και 12 εκ των οποίων οι δύο
πρώτοι συνδέονται στο δίαυλο SPI ενώ ο τελευταίος στην τροφοδοσία 3.3V της
πλακέτας. Οι υπόλοιποι, με εξαίρεση τον 20 (\textenglish{Not Connected}),
συνδέονται με το σημείο αναφοράς, δεδομένου ότι χρησιμοποιούνται μόνο στην
περίπτωση άμεσης ή έμμεσης προσπέλασης.
>>>>GND

Header J2
Ο ακροδέκτης 1 συνδέεται στην τροφοδοσία 3.3V της πλακέτας, οι 3 και 4 στο
δίαυλο SPI, ενώ οι 2 και 8 σε ελεύθερους ακροδέκτες της πλακέτας.
Οι 5 έως 7 χρησιμοποιούνται στην περίπτωση άμεσης ή έμμεσης προσπέλασης και όχι
σε SPI. Ωστόσο, είναι active low και, επομένως, συνδέονται μόνιμα, μέσω
αντιστάτη, με τάση 3.3V ώστε να είναι λογικά ανενεργοί.

Εφόσον το W5100 ρυθμιστεί, μπορεί να ειδοποιεί την MCU 

Ο ακροδέκτης 8 μπορεί να χρησιμοποιηθεί για να ειδοποιείται η MCU ότι υπάρχει
λόγος να επικοινωνήσουν, πχ επειδή έχουν καταφθάσει δεδομένα, και παραμένει σε
λογικό 0 έως ότου η MCU αλλάξει την τιμή των καταχωρητών του W5100. Διαφορετικά,
η MCU πρέπει να ===POLL-SELECT===

Βέβαια, η χρήση του ακροδέκτη INT προϋποθέτει ότι η MCU μπορεί και έχει
ρυθμιστεί ώστε να ανταποκρίνεται σε εξωτερικές διακοπές στο συνδεδεμένο
ακροδέκτη.

Υποστηρίζεται άμεση ή έμμεση προσπέλαση με χρήση διαύλου διευθύνσεων, δεδομένων
και ελέγχου ή, εναλλακτικά, μέσω πρωτοκόλλου SPI.

Οι δύο πρώτες μέθοδοι απαιτούν πολλά σημεία σύνδεσης με την MCU. Για την
ακρίβεια, 4 γραμμές ελέγχου (Chip Select, Read, Write, Interrupt), 8 για το
δίαυλο δεδομένων, και, 15 ή 2 γραμμές διεύθυνσης για άμεση ή έμμεση προσπέλαση,
αντίστοιχα.
Στην περίπτωση του SPI απαιτούνται πολύ λιγότερες (μόλις 4) και αυτός είναι ο
λόγος που προτιμάται για τη διασύνδεση του με την MCU, δεδομένου του περιορισμένου   ~~~~
αριθμού ακροδεκτών της. Σαφώς, το μειονέκτημα χρήσης SPI -- ενός σειριακού
πρωτοκόλλου επικοινωνίας -- είναι ότι επιτυγχάνεται πολύ μικρότερος ρυθμός
ανταλλαγής δεδομένων από ότι στην περίπτωση των άλλων δύο που, τελικά, επηρεάζει
το χρόνο απόκρισης στα εισερχόμενα αιτήματα. Ωστόσο, κρίνεται ότι για τις
ανάγκες της υλοποίησης, αυτοί οι περιορισμοί είναι αμελητέοι.

~\\\nref{} Παραπομπή σε SPI.

Λόγω περιορισμένου αριθμού ακροδεκτών της MCU, προτιμάται η διασύνδεση μέσω SPI.


\parencite[95]{atmel13}\\
Ο μετρητής TCNT0 υποστηρίζει εσωτερικό χρονισμό καθώς και εξωτερικό από τον
ακροδέκτη T0. bit (CS02:0 -- Clock Select) και καταχωρητής (TCCR0B -- Timer
Counter Control Register) που ελέγχουν τη συμπεριφορά.

\parencite[96]{atmel13}\\
clear, increment or decrement at each timer clock (clk$_{\text{T0}}$)
timer stop with no selected clk source
CPU has priority over all counter clear/count operations.

συνεχής σύγκριση TCNT0 και Output Compare Register (OCR0A και OCR0B).
Όταν οι τιμή του πρώτου ταυτίζεται με την τιμή κάποιου από τους τελευταίους,
μπορούν να συμβούν τα ακόλουθα:
1) Εξαίρεση (εφόσον έχει ενεργοποιηθεί, ο ενδείκτης ούτως ή άλλως τίθεται).
 Output Compare Flag (OCF0A ή OCF0B) και, εφόσον έχουν ενεργοποιηθεί, καλείται
η ρουτίνα εξυπηρέτησης της διακοπής. Ο ενδείκτης καθαρίζεται αυτόματα με την
κλήση της ρουτίνας ή, εναλλακτικά, από το λογισμικό, γράφοντας λογικό 1.
2) Δημιουργία παλμού. Waveform Generator (Γεννήτρια κυματομορφής) σύμφωνα με τις
τιμές που έχουν δοθεί στα bit WGM02:0 και COM0x1:0 (x={A,B}?).

15.5.2 - Δυνατότητα αρχικοποίησησ OCR0x σε οποιαδήποτε τιμή (ακόμα και ίσης με
του TCNT0).


15.6 (και σχήμα 15-4) Τα COM02:0 ελέγχουν ...
Η τιμή της Γεννήτριας υπερισχύει έναντι της τιμής που έχει τεθεί για αυτόν τον
ακροδέκτη. Ωστόσο, η κατεύθυνση του (είσοδος ή έξοδος) πρέπει να έχει τεθεί
εξαρχής ως έξοδο.

\begin{flushleft}
OC2A (PB3), OC2B (PD3) of TCNT2 → 8-bit PWM (phase corrent, but not frequency)\\
T0 (PD4) of TCNT0 and T1 (PD5) of TCNT1 as 256-step counter
\end{flushleft}

~\\module\\
~\\master, slave (SPI)\\
~\\active low\\
~\\connector?/συνδετήρας (RJ-45)\\
~\\header (connector)\\
~\\LaTeX active low symbol\\

%Αντιστοίχιση όρων
%PWM	διαμόρφωση διάρκειας παλμών	Βιβλίο τηλεπικοινωνιών

%optocoupler	οπτικός συζεύκτης	 Reg. 1572/93 (1) OJ L 156/93 p.71
