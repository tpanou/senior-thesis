\chapter{Κατασκευή}
\label{ch:construction}

Το κεφάλαιο ασχολείται με την επιλογή και συνδυασμό μηχανικών εξαρτημάτων
προκειμένου να συντεθεί η φυσική υπόσταση της συσκευής (\te{hardware}), η οποία
προορίζεται να ελέγχεται από το λογισμικό του μικροελεγκτή.

Το σημαντικότερο στοιχείο της συσκευής είναι η κεφαλή, η οποία πρόκειται για ένα
όργανο που κινείται σε επίπεδο πάνω από το παρακολουθούμενο υλικό και κατακόρυφα
προς αυτό εισχωρώντας το, ώστε τα αισθητήρια όργανα που φέρει να έρχονται σε
επαφή με το υλικό και να πραγματοποιούν μετρήσεις των συνθηκών που επικρατούν
σε κάθε σημείο.

Κρίνεται ότι ο συνδυασμός κινητής κεφαλής και αισθητήρων έχει το πλεονέκτημα
χρήσης μικρού αριθμού αισθητήρων καθώς η κινητικότητά τους τους καθιστά ικανούς
να καλύπτουν όλο το υλικό. Επίσης, σε ενδεχόμενη ανάγκη παρακολούθησης υλικού
που καταλαμβάνει ένα μικρό μέρος της μέγιστης υποστηριζόμενης επιφάνειας, η
οποιαδήποτε ρύθμιση αρκεί να γίνεται μέσα από το λογισμικό της συσκευής ώστε,
απλώς, να περιορίζεται η επιφάνεια που καλύπτει.
Ωστόσο, απαιτείται μηχανισμός που υποστηρίζει τη γραμμική κίνηση της κεφαλής στο
χώρο καθώς και υποδομή που επιτρέπει τη στήριξη των σχετικών εξαρτημάτων.

Αρχικά, μελετώνται τα συστήματα CNC (\te{Computer Numerical Control})·
εργαλειομηχανές (όπως τόρνος, γραναζοκόπτης, τρισδιάστατος εκτυπωτής) των οποίων
ο χειρισμός γίνεται μέσω υπολογιστή παρέχοντας πολύ μεγαλύτερη ταχύτητα και
ακρίβεια στην εργασία (\nameref{sec:construct:cnc} σ.~%
\pageref{sec:construct:cnc}). Ωστόσο, όπως αναφέρεται και στη σχετική ενότητα, η
μελέτη τους γίνεται, όχι τόσο για την κατανόηση του τρόπου λειτουργίας τους,
αλλά για τον εντοπισμό μίας κατάλληλης δόμησης της συσκευής που, όπως φαίνεται
στο σχήμα \ref{fig:construct:device}, ακολουθεί διάταξη κινητού γεφυρώματος.

Τα στοιχειώδη δομικά στοιχεία (οδηγοί\slash{}ράγες, πλάκες συνένωσης) παρέχονται
από σύστημα κατασκευής (\te{construction framework}) το οποίο επιλέγεται με
γνώμονα την παροχή ευελιξίας κατά την ανάπτυξη του αρχετύπου ιδίως με
ανασχεδιασμό μερών της συσκευής καθώς η υλοποίησή της βρίσκεται ακόμα σε
εξέλιξη. Τα βασικά στοιχεία του επιλεγμένου συστήματος κατασκευής, Open\-Builds,
αναφέρονται στη σελίδα \pageref{sec:construct:framework}.

Στην πορεία, ακολουθούν τα βασικά μέρη που απαρτίζουν τη συσκευής και πώς αυτά
συντίθενται από τα διαθέσιμα δομικά στοιχεία OpenBuilds. Αρχικά, αναφέρεται η
βάση της συσκευής, η οποία πρόκειται για το πλαίσιο πάνω στο οποίο
πραγματοποιείται όλη η κίνηση (σ.~\pageref{sec:construct:base}). Ακολουθούν οι
οδηγοί κίνησης (άξονες X, Y και Z) (σ.~\pageref{sec:construct:axes}) και,
τελικά, η τοποθέτηση των κινητήρων (σ.~\pageref{sec:construct:motors}).

\begin{figure}
    \caption{Αναπαράσταση της συσκευής.\label{fig:construct:device}}
    \begin{center}%
    \includegraphics[width=\textwidth]{construct_device_2.png}
    \end{center}
\end{figure}

\section{Μηχανή CNC}
\label{sec:construct:cnc}

Μηχανή αριθμητικού ελέγχου (Numerical Control -- NC)\index{αριθμητικός
έλεγχος, NC} είναι μία μηχανή η οποία δέχεται κωδικοποιημένες οδηγίες για την
εκτέλεση των επιθυμητών ενεργειών και, κατά βάση, αναφέρεται στον έλεγχο των
εξαρτημάτων της συσκευής παρά σε συγκεκριμένου τύπου μηχανή
\parencites{seames01}{albert11}.

Αρχικά, οι εντολές δίνονταν στη μηχανή μέσω διάτρητων καρτών ή χαρτοταινιών,
ωστόσο, η εξέλιξη των μικροϋπολογιστών επέτρεψε τη χρήση τους για την εκτέλεση
του προγράμματος (Computer Numerical Control - CNC)\index{αριθμητικός
έλεγχος με υπολογιστή, CNC} \parencite{seames01}.

Δεδομένου του ορισμού μία μηχανής CNC, η συσκευή δεν μπορεί να θεωρηθεί ως
τέτοια. Βασικός λόγος είναι ότι η συσκευή αυτοματοποιεί μία διαδικασία δεχόμενη
κάποιες παραμέτρους που ρυθμίζουν ελαφρώς ορισμένες λειτουργίες της χωρίς,
ωστόσο, να απαιτείται η παροχή ενός συνόλου οδηγιών οι οποίες ελέγχουν τα μέρη
της για την πραγματοποίηση της επιθυμητής εργασίας, όπως δηλαδή, απαιτείται από
μηχανές CNC.
Ωστόσο, κρίνεται χρήσιμη η γνωριμία με τέτοια συστήματα καθώς, ακολουθώντας μία
εξελικτική πορεία από το 1952, έχει συσσωρευτεί εμπειρία η οποία μπορεί να φανεί
χρήσιμη για τις ανάγκες της συσκευής.

Ένα από τα ζητήματα που έχουν κληθεί να αντιμετωπίσουν είναι η κίνηση των
εξαρτημάτων τους στο χώρου. Έχουν αναπτυχθεί διάφορες για το σκοπό αυτό οι
οποίες μελετώνται για την ανάδειξη κάποιας κατάλληλης για την υλοποίηση.


\subsection{Διατάξεις CNC}

Η τράπεζα ενός CNC είναι η επιφάνεια πάνω στη οποία εναποτίθεται το προς
επεξεργασία υλικό του συστήματος η οποία είναι δυνατό να κινείται προκειμένου να
εξυπηρετεί κίνηση σε κάποιον άξονα ή να είναι πλήρως σταθερή. Βάσει αυτού του
χαρακτηριστικού, οι διατάξεις CNC διακρίνονται σε κινητής και σταθερής τραπέζης.

Πέραν των διατάξεων που αναφέρονται παρακάτω, υφίστανται κι άλλες που βρίσκουν
εφαρμογή στην
επεξεργασία μεγάλων τρισδιάστατων αντικειμένων, όπως διατάξεις 5 διαστάσεων ή
βιομηχανικά ρομπότ, υποστηρίζοντας περιστροφική κίνηση γύρω από άξονες
επιπροσθέτως της ευθύγραμμης κίνησης σε αυτούς, οι οποίες ξεπερνούν δραματικά
τις απαιτήσεις της υλοποίησης και, συνεπώς, αποκλείονται από αυτήν.

\subsubsection{Κινητή τράπεζα}

\begin{figure}
    \caption{Διατάξεις κινητής τραπέζης.
        \label{fig:construct:cnc_moving-table}}
    \begin{center}
        \begin{subfigure}[b]{0.30\textwidth}
            \includegraphics[width=0.95\textwidth]{construct_cnc_xy-table.png}
            \caption{}
        \end{subfigure}
        \begin{subfigure}[b]{0.45\textwidth}
            \includegraphics[width=0.9\textwidth]{construct_cnc_moving-table.png}
            \caption{}
        \end{subfigure}
    \end{center}

    (αʹ): \fullcite{albert11:xy-table}

    (βʹ): \fullcite{albert11:moving-table}
\end{figure}

Αναφέρονται δύο διατάξεις που χρησιμοποιούν κινητή τράπεζα. Σε κάθε περίπτωση, η
κίνηση στον κατακόρυφο προς την τράπεζα άξονα, Z, εκτελείται από σταθερό σημείο.
Στην πρώτη διάταξη, η τράπεζα κινείται τόσο στον άξονα X όσο και στον άξονα Y,
με την κίνηση στον άξονα Z να εκτελείται από σταθερή δοκό που εκτείνεται πάνω
από την τράπεζα, στερεωμένη στη βάση μέσω κατακόρυφης στήλης (σχήμα
\ref{fig:construct:cnc_xy-table}α) \parencite[69]{albert11}.
Η τράπεζα αυτής της διάταξης περιορίζεται σε, σχετικά, μικρές διαστάσεις καθώς,
παράλληλα με αυτές, αυξάνει και το μήκος της δοκού και οι απαιτήσεις για τη
στερέωσή της.

Σε εναλλακτική διάταξη, η κίνηση στο επίπεδο X-Y διαμοιράζεται μεταξύ της
τραπέζης και μίας κεφαλής η οποία κινείται στον έναν εκ των δύο αξόνων πάνω σε
γεφύρωμα που διατρέχει όλο το πλάτος της συσκευής (σχήμα
\ref{fig:construct:cnc_xy-table}β) \parencite[70]{albert11}.

Οι διατάξεις κινητής τραπέζης έχουν το μειονέκτημα ότι η κίνηση της τραπέζης
επιβαρύνεται από το υπό παρακολούθηση υλικό θέτοντας περιορισμούς στο μέγιστο
επιτρεπτό βάρος του φορτίου.
Ένα δεύτερο μειονέκτημα είναι ότι η βάση της συσκευής πρέπει να είναι αρκετά
μεγαλύτερη από την τράπεζα μόνο και μόνο για την εξυπηρέτηση της κίνησης με
αποτέλεσμα να δεσμεύεται επιπρόσθετος χώρος χωρίς αυτός να αξιοποιείται
ουσιαστικά για τις ανάγκες της υποβοηθούμενης διαδικασίας.


\subsubsection{Σταθερή τράπεζα}

Σε μία πρώτη διάταξη σταθερής τραπέζης χρησιμοποιείται κινητή στήλη στήριξης
μίας, επίσης, κινητής δοκού, έκαστη κινούμενη σε διαφορετικό άξονα (σχήμα
\ref{fig:construct:cnc_fixed-table}α) \parencite[70]{albert11}. Άμεση πρόκληση
αυτής της διάταξης, η οποία απορρέει από το ότι η δοκός στηρίζεται μόνο σε μία
πλευρά, σχετίζεται με τη σταθερότητα της δοκού ιδίως με την επιμήκυνση της για
την κάλυψη μεγαλύτερης επιφάνειας.

Εναλλακτική διάταξη μοιάζει με τη δεύτερη διάταξη κινούμενης τραπέζης που έχει
αναφερθεί με τη διαφορά ότι αντί για την τράπεζα, κινείται το γεφύρωμα (σχήμα
\ref{fig:construct:cnc_fixed-table}β) \parencite[71]{albert11}.
Ένα ενδεχόμενο μειονέκτημα αυτής της περίπτωσης είναι η ανάγκη για ύπαρξη
εξαρτημάτων κίνησης σε κάθε άκρο του γεφυρώματος.

\begin{figure}
    \caption{Διατάξεις σταθερής τραπέζης.
        \label{fig:construct:cnc_fixed-table}}
    \begin{center}
        \begin{subfigure}[b]{0.40\textwidth}
            \includegraphics[width=0.95\textwidth]{construct_cnc_cantilevered.png}
            \caption{}
        \end{subfigure}
        \begin{subfigure}[b]{0.40\textwidth}
            \includegraphics[width=0.9\textwidth]{construct_cnc_moving-gantry.png}
            \caption{}
        \end{subfigure}
    \end{center}

    (αʹ): \fullcite{albert11:cantilevered}

    (βʹ): \fullcite{albert11:moving-gantry}
\end{figure}


\subsection{Υπόδειγμα κατασκευής}

Μεταξύ των δύο κατηγοριών διατάξεων που έχουν αναφερθεί, προτιμώνται διατάξεις
σταθερής τραπέζης κι αυτό επειδή, αφενός, αξιοποιούν καλύτερα το χώρο που τους
αφιερώνεται, αφετέρου επηρεάζουν και επηρεάζονται λιγότερο από τα χαρακτηριστικά
του δοχείου.
Σε σχέση με το δεύτερο, κρίνεται αναπόφευκτη η εξάρτηση των διαστάσεων του
δοχείου από τις διαστάσεις της συσκευής, δεδομένου ότι η συσκευή το πλαισιώνει
και κινείται γύρω από αυτό, ανεξαρτήτως της επιλεγμένης διάταξης. Ωστόσο, η μάζα
του φορτίου, η οποία εξαρτάται από τη μάζα των περιεχόμενων υλικών καθώς και την
απορροφητικότητα τους και την εκάστοτε υγρασία που επικρατεί στο δοχείο, χρήζει
μελέτης μόνο στην περίπτωση των διατάξεων κινητής τραπέζης. Επιλέγοντας
διατάξεις που διατηρούν μονίμως σταθερή την τράπεζα, εξαλείφονται ενδεχόμενοι
περιορισμοί στο μέγιστο επιτρεπτό φορτίο.
%
%Ωστόσο, το μέγιστο επιτρεπτό βάρος του δοχείου, το οποίο επηρεάζεται από τη μάζα
%των υλικών και την απορροφητικότητα τους
%
%Τα όρια, μέγιστα και ελάχιστα, των διαστάσεων του δοχείου υπόκεινται σε
%πρακτικούς περιορισμούς που σχετίζονται, κυρίως, με τη φυσιολογία του πληθυσμού.
%
%Επίσης, θεωρείται καλύτερη πρακτική η δυνατότητα επέκτασης της συσκευής ώστε να
%φιλοξενεί περισσότερη ποσότητα υλικών, χωρίς να απαιτείται αναπροσαρμογή μίας
%τόσο βασικής λειτουργίας της, αυτής της κίνησης.

Επόμενο στοιχείο που μελετάται είναι η αναγκαιότητα ύπαρξης τραπέζης. Δεδομένου
ότι στο δοχείο επικρατούν συνθήκες υψηλής υγρασίας και η κατασκευή αποτελείται
κυρίως από μέταλλο, κρίνεται σκόπιμο τα δύο να είναι πλήρως ανεξάρτητα.
Παράδειγμα αποτελεί η αποστράγγιση, η οποία αν παρέχεται από το δοχείο είναι
προτιμότερο να μην έρχεται σε επαφή με τη συσκευή.
Με αυτόν τον τρόπο, αποφεύγεται η οξείδωση μερών της συσκευής ενώ παράλληλα
επιτρέπεται η ελεύθερη επιλογή, διαμόρφωση και αντικατάσταση του δοχείου χωρίς
να επηρεάζεται η ίδια η συσκευή.
Η ανεξαρτησία του δοχείου σε σχέση με τη συσκευή σε συνδυασμό με το ότι δεν
είναι απαραίτητη η πρόσδεση του για την κίνηση, αποτελούν τους βασικούς λόγους
για την περαιτέρω απάλειψη της τραπέζης.

Σε σχέση με τις δύο προαναφερθείσες διατάξεις σταθερής τραπέζης, επιλέγεται αυτή
του κινητού γεφυρώματος επειδή παρέχει σταθερότητα με περισσότερη ευκολία.


\section{Σύστημα κατασκευής}
\label{sec:construct:framework}

Κρίνεται απαραίτητη η εύρεση και αξιοποίηση ενός συστήματος κατασκευής αρχετύπων
που επιταχύνει τη διαδικασία κατασκευής παρέχοντας ευελιξία στην τροποποίηση της
υλοποίησης καθώς αυτή αναπτύσσεται και προκύπτουν νέες απαιτήσεις ή και
προβλήματα, η οποία, ενδεχομένως, παρέχει λύσεις σε συχνές ανάγκες στο πλαίσιο
κατασκευής.

Εντοπίστηκαν αρκετά συστήματα (ουσιαστικά το 80/20 και παραλλαγές αυτού όπως
Misumi, MakerBeam) τα οποία χρησιμοποιούν ράβδους εξωθημένου αλουμινίου ως
δομικά στοιχεία τα οποία συνδυάζονται μεταξύ τους μέσω προσαρτημάτων για την
κατασκευή του σκελετού της κατασκευής.

Η εξώθηση αλουμινίου είναι μία τεχνολογία πλαστικής, δηλαδή μόνιμης,
παραμόρφωσης όπου αλουμίνιο εξωθείται να διέλθει μέσω ενός ανοίγματος
συγκεκριμένου σχήματος για την δημιουργία ράβδων αντίστοιχης διατομής, και τα
προϊόντα της βρίσκουν εφαρμογή σε διάφορους τομείς όπως στην αρχιτεκτονική,
τη βιομηχανία αυτοκινήτων και την παραγωγή μικρών μηχανικών και δομικών
στοιχείων \parencite{saha00}.

Το θετικό αυτής της λύσης είναι ότι χρησιμοποιούνται ξεχωριστά προσαρτήματα για
τη στερέωση των ράβδων και όχι μόνιμη οξυγονοκόλληση επιτρέποντας την ανά πάσα
στιγμή αναδιαμόρφωση των συνδέσεων ενόψει νέων απαιτήσεων ή προβλημάτων και,
λόγω της μορφής τους, επιτρέπουν την ενασχόληση με μεμονωμένα τμήματα με την
περαιτέρω σύνθεσή τους στην τελική κατασκευή.
Για την εξυπηρέτηση ευθύγραμμης κίνησης, οι ράβδοι χρησιμοποιούνται σε συνδυασμό
με τροχοφόρες διατάξεις ή διατάξεις που φέρουν ένσφαιρους τριβείς των ιδίων ή
διαφορετικών συστημάτων.

Επιπροσθέτως αυτών, εντοπίστηκαν συστήματα που αφοσιώνονται εξ ολοκλήρου στη
γραμμική κίνηση, όπως Thomson Linear, Rexroth Bosch Group, NSK Linear.
Μολονότι τα εν λόγω συστήματα, ενδεχομένως, χρησιμοποιούνται κατά κόρον σε
βιομηχανικές εφαρμογές, οι απαιτήσεις αυτής της εφαρμογής είναι αρκετά
διαφορετικές. Ενδιαφέρει περισσότερο η ευελιξία κατά την υλοποίηση (όπως η
δυνατότητα χρήσης οποιασδήποτε δοκού ως οδηγό κίνησης), η ευκολία αναπροσαρμογής
των στοιχείων (για παράδειγμα, χρήση αλουμινίου αντί του πιο σκληρού χάλυβα),
και οι χαμηλές απαιτήσεις συντήρησης, όπως ελάχιστη ή καθόλου ανάγκη για λίπανση
εξαρτημάτων (για παράδεγιμα, για την επιμήκυνση της διάρκειας ζωής τους), και
όλα αυτά, ακόμα και εις βάρους της συνολικής απόδοσης της κατασκευής ή ενός
λιγότερου κομψού αποτελέσματος.

%Επιπλέον, δεδομένων των εγγενώς χαμηλών απαιτήσεων της τρέχουσας εφαρμογής,

Το επιλεγμένο σύστημα διαθέτει όλα τα προαναφερθέντα χαρακτηριστικά. Ένας
επιπρόσθετος λόγος για την επιλογή αυτού έναντι αντίστοιχων συστημάτων είναι
ότι παρέχει μία ποικίλη συλλογή δομικών στοιχείων χωρίς, ωστόσο, να είναι
αχανής, ικανή να καλύψει πληθώρα αναγκών. Επιπλέον, υπερτερεί στο κομμάτι του
αναμενόμενου χρόνου ζωής των τροχών, ενός αναμφισβήτητα σημαντικού δομικού
στοιχείου αυτών των συστημάτων. Τέλος, τα στοιχεία παρέχονται ως ανοικτό υλικό
με ελεύθερη τη λήψη αρχείων για την ενσωμάτωσή τους σε εφαρμογές σχεδίασης 3-Δ.

%Προτιμάται η χρήση τροχοφόρων διατάξεων έναντι ένσφαιρων τριβέων ευθύγραμμης
%κίνησης επειδή οι τριβείς συνήθως απαιτούν λίπανση και περιβάλλον ελεύθερο από
%ρύπους για τη μεγιστοποίηση της απόδοσης και της διάρκειας ζωής τους (REF).

Στη συνέχεια παρουσιάζεται ο τρόπος ενσωμάτωσης δομικών στοιχείων του συστήματος
OpenBuilds για τις ανάγκες της υλοποίησης.


\subsection{Οδηγοί}

\begin{figure}
    \caption{Τα τέσσερα μεγέθη οδηγών V-Slot.
    \label{fig:construct:vslot}}
    \begin{center}%
    \def\svgwidth{0.8\textwidth}
    \input{img/construct_vslot.pdf_tex}
    \end{center}

    Βασισμένο \fullcite{carew:vslot}
\end{figure}

Το βασικότερο δομικό στοιχείο του συστήματος είναι το VSlot, εφεξής οδηγός, το
οποίο
χρησιμοποιείται τόσο για την κατασκευή του σκελετού όσο και για γραμμική κίνηση,
επιτρέποντας οποιοδήποτε τμήμα της κατασκευής να χρησιμοποιηθεί ως φορέας
κινητών φορτίων.

Οι οδηγοί διαθέτουν αυλακώσεις οι οποίες, επικλινείς στο ανώτερο τμήμα,
χρησιμοποιούνται ως διάδρομοι τροχών και ως υποδοχείς ιμάντα, καλωδίων ή άλλων
συνδετικών εξαρτημάτων στο κατώτερο τμήμα.
Στο σχήμα \ref{fig:construct:vslot} παρατίθεται η μορφή των τεσσάρων διαθέσιμων
μεγεθών οδηγών.
Κατασκευάζονται σε μήκος 1 και 1.5m από κράμα αλουμινίου 6063-T5 -- μέταλλο
μαλακό -- το οποίο επιτρέπει την εύκολη αναπροσαρμογή του μήκους τους.


\subsection{Τροχοί}

\begin{figure}
    \caption{Μέλη που αποτελούν τον τροχό.\label{fig:construct:wheel_exploded}}
    \begin{center}%
    \def\svgwidth{0.8\textwidth}
    \input{img/construct_wheel_exploded.pdf_tex}
    \end{center}
\end{figure}

Συντίθενται από ανεξάρτητα μέρη, με αυτόν τον τρόπο επιτρέποντας την προσαρμογή
τους στις απαιτήσεις κάθε περίπτωσης, την αντικατάσταση κάποιου πιθανού
ελαττωματικού μέρους αντί ολόκληρου του τροχού καθώς και πιθανές μελλοντικές
επεκτάσεις της κατασκευής τους.
Το επίσωτρο των τροχών διαθέτει, σαφώς, συμβατή μορφή με τις αυλακώσεις των
οδηγών και εφάπτεται στο ανώτερο τμήμα των αυλακώσεων δίχως να εισέρχεται πλήρως
μέσα σε αυτές ώστε να αποφεύγονται φθορές στον τροχό κατά την κίνησή του από την
επαφή του με τα τοιχώματα.
%Στην εσωτερική κοιλότητα του επίσωτρου τοποθετείται, εφαρμοστά, ζεύγος ένσφαιρων
%τριβέων για τη μείωση των τριβών μεταξύ του τροχού και του άξονα περιστροφής.
%Στο σχήμα \ref{fig:construct:wheel_exploded} παρατίθενται τα μέρη που συνθέτουν
%έναν υποδειγματικό τροχό· βίδα M5, διαχωριστικό, ένσφαιρος τριβέας, επίσωτρο,
%δακτύλιος, δεύτερος ένσφαιρος τριβέας, αντιπερικόχλιο. Ο λόγος ύπαρξης του
%έκκεντρου διαχωριστικού (eccentric spacer) αιτιολογείται σε επόμενη παράγραφο.
% ως άξονας περιστροφής και για την πρόσδεση του τροχού

Στο σχήμα \ref{fig:construct:wheel_exploded} παρατίθενται τα μέρη που συνθέτουν
έναν υποδειγματικό τροχό, και, συνοπτικά περιγράφονται παρακάτω:
\begin{flushleft}
\begin{description}
    \item[Βίδα M5 (α)] ως άξονας περιστροφής και για την στερέωση του τροχού.
    \item[Διαχωριστικό (β)] για την κάλυψη της απόστασης μέχρι την αυλάκωση.
    \item[Έκκεντρο διαχωριστικό (γ)] ως εναλλακτικό του απλού διαχωριστικού. Η
    χρήση αυτού αντί του απλού διαχωριστικού περιγράφεται σε επόμενη παράγραφο.
    \item[Ένσφαιροι τριβείς (δ)] για την μείωση των τριβών μεταξύ τροχού και
    άξονα περιστροφής.
    \item[Επίσωτρο (ε)] για την επαφή με την αυλάκωση.
    \item[Δακτύλιος (στ)] ενδιάμεσος των τριβέων, για την αποφυγή συμπλοκής
    τους.
    \item[Αντιπερικόχλιο (ζ)] για τη συγκράτηση του τροχού στον άξονα.
\end{description}
\end{flushleft}

Οι τροχοί στερεώνονται απευθείας στις αυλακώσεις κάποιου οδηγού δίνοντάς του τη
δυνατότητα κύλισης ή σε ξεχωριστή διάτρητη πλάκα η οποία μπορεί να φιλοξενήσει
και πληθώρα άλλων εξαρτημάτων. Η βασικότερη πλάκα είναι η Universal Gantry
Plate, εφεξής πλάκα γεφυρώματος, το αρχέτυπο της οποίας απεικονίζεται στο σχήμα
\ref{fig:construct:gantry-plate}. Οι περισσότερες κυκλικές οπές της είναι
διαμέτρου 5mm ενώ οι επιμήκεις μπορούν να χρησιμοποιηθούν για την πρόσδεση
ιμάντων.

\begin{figure}
    \caption[Πλάκα γεφυρώματος]{Πλάκα γεφυρώματος (μονάδες σε mm).
    \label{fig:construct:gantry-plate}}
    \begin{center}%
    \includegraphics[width=\textwidth]{construct_gantry-plate.png}%
    \end{center}

    \fullcite{carew:gantry-plate-schematic}
\end{figure}

Ορισμένες οπές είναι διαμέτρου 7.14mm και βρίσκονται συγκεντρωμένες στο δεξί
μισό του σχήματος.
Οι συγκεκριμένες χρησιμοποιούνται σε συνδυασμό με έκκεντρα διαχωριστικά έναντι
απλών (σχήμα \ref{fig:construct:wheel_exploded}γ) των οποίων η προεξοχή
εισέρχεται στην οπή. Περιστρέφοντας τους, μεταβάλλεται η απόσταση των
αντίστοιχων τροχών από τον οδηγό έως ότου εφάπτονται ερμητικά με την αυλάκωση
ώστε να αποφεύγονται οι κραδασμοί κατά την κίνηση της πλάκας.
Τοποθετώντας τροχούς σε κατάλληλα ζεύγη οπών 5 και 7.14mm δημιουργείται διάκενο
στο οποίο εισέρχεται οποιοδήποτε από τα τέσσερα μεγέθη οδηγών.


%\section{Γραμμική κίνηση}


\section{Βάση}
\label{sec:construct:base}

\begin{figure}
    \caption{Η βάση της συσκευής. \label{fig:construct:base}}
    \begin{center}%
    \def\svgwidth{\textwidth}
    \input{img/construct_base_exploded.pdf_tex}
    \end{center}
\end{figure}

Η βάση της συσκευής αποτελείται από πλαίσιο κατασκευασμένο από παραλληλεπίπεδους
οδηγούς, αναρτημένο σε τέσσερις γωνιακούς οδηγούς στήριξης. Η συγκράτηση των
οδηγών επιτυγχάνεται με χρήση γωνιακών προσαρτημάτων, κοχλιών M5 και περικοχλίων
ένθετων στις αυλακώσεις των οδηγών (σχήμα \ref{fig:construct:base}). Όπως
γίνεται αντιληπτό, στοιχεία όπως κοχλίες, περικόχλια, αντιπερικόχλια, δακτύλιοι
\etc{.} είναι απαραίτητα σε κάθε στάδιο της κατασκευής. Ωστόσο, σημειώνεται ότι
αποφεύγεται η τόσο λεπτομερής ανάδειξη των επιμέρους αυτών στοιχείων, εφόσον
κρίνεται ότι στερούνται ουσιαστικού ενδιαφέροντος.

Στον κενό χώρο που σχηματίζεται κάτω από το πλαίσιο, τοποθετείται το δοχείο
παρακολούθησης το οποίο και τον καλύπτει χωρίς, ωστόσο, να εισέρχεται στο
πλαίσιο. Ο χώρος του πλαισίου αφιερώνεται, εξ ολοκλήρου, στην κίνηση των οργάνων
της συσκευής και είναι απαραίτητο να παραμένει ελεύθερος από εμπόδια. Αυτός
είναι και ο λόγος που έχουν επιλεγεί για το πλαίσιο οι τόσο πλατιοί οδηγοί των
8cm· ώστε να καλύπτονται επαρκώς τα κινητά του όργανα καθώς και να παρέχεται ένα
εύληπτο όριο για το ύψος του δοχείου. Περισσότερες λεπτομέρειες σχετικά με τα
κινητά όργανα που προστατεύονται από το πλαίσιο δίνονται στην ενότητα
\nameref{sec:construct:z-axis} (σ.~\pageref{sec:construct:z-axis}).

Το ύψος της βάσης και, συνεπώς, του δοχείου, περιορίζεται από το μέγεθος των
ίδιων των αισθητήριων οργάνων καθώς θα ήταν άσκοπο έως και επιβαρυντικό για την
εξαγωγή έγκυρων μετρήσεων, εάν υπάρχει βάθος στο δοχείο το οποίο παραμένει
απροσπέλαστο από τους αισθητήρες και αδύνατο να καταμετρηθεί.
Δεδομένου ότι οι επιλεγμένοι αισθητήρες υγρασίας και οξύτητας έχουν μήκος
περίπου 21cm, για τη στήριξη του πλαισίου επιλέγονται οδηγοί 30cm, 8cm των
οποίων χρησιμοποιούνται για την προσάρτησή τους στο πλαίσιο (σχήμα
\ref{fig:construct:base}).

Οι άλλες δύο διαστάσεις είναι σχεδόν ανεξάρτητες με τους μοναδικούς περιορισμούς
να τίθενται σχεδόν κατά αποκλειστικότητα από τις δυνατότητες του
μικροεπεξεργαστή καθώς και από πρακτικούς λόγους όπως το μέγεθος της επιφάνειας
προς διαχείριση.
Ωστόσο, επειδή έχει γίνει ήδη αναφορά στο μέγεθος του δοχείου σε σχέση με τη
βάση, σημειώνεται ότι η υπό παρακολούθηση περιοχή είναι ελαφρώς μικρότερη από
τις πραγματικές εσωτερικές διαστάσεις της βάσης εξαιτίας ενός περιθωρίου μερικών
εκατοστών -- μίας περιοχής απροσπέλαστης από τα αισθητήρια όργανα.
Οι λόγοι ύπαρξης του περιθωρίου γίνονται αντιληπτοί στις επόμενες ενότητες.


\section{Άξονες κίνησης}
\label{sec:construct:axes}


\subsection{Άξονας Y}
\label{subsec:construct:y}

Τροχοφόρος πλάκα τοποθετείται παράλληλα προς το έδαφος με τους τροχούς της στην
κορυφαία αυλάκωση ενός περιμετρικού οδηγού της βάσης. Στο σχήμα
\ref{fig:construct:belt-pulley-y} απεικονίζεται η βασική διάταξη των εξαρτημάτων
σε μία ενδεικτική απλοποιημένη υλοποίηση.
Στην εξωτερική πλευρά της πλάκας (α) βρίσκεται κινητήρας από τον οποίο
εκτείνεται ράβδος (β) που φέρει την κινητήρια τροχαλία (γ) (ο κινητήρας
στερεώνεται πάνω στην πλάκα, ωστόσο, έχει αποκρυφτεί από το συγκεκριμένο σχήμα).
Τραπεζοειδής ιμάντας τεντωμένος κατά μήκος της εξωτερικής αυλάκωσης του οδηγού
και στερεωμένος στα άκρα του, αξιοποιεί τους τροχούς ως ελεύθερες τροχαλίες ώστε
να αυξάνεται το τόξο επαφής με την κινητήρια τροχαλία (γ), παρέχοντας μεγαλύτερη
σταθερότητα κατά την κίνηση.

\begin{figure}
    \caption{Κινητή τροχαλία και ιμάντας. \label{fig:construct:belt-pulley-y}}
Διάταξη τροχαλίας-ιμάντα για την κίνηση στον άξονα Y. Ο κινητήρας έχει
αποκλειστεί από την απεικόνιση. Ωστόσο, νοείται ότι βρίσκεται συζευγμένος με τη
ράβδο (β) στο ελεύθερο άκρο της.
    \begin{center}%
    \def\svgwidth{0.7\textwidth}
    \input{img/construct_belt-pulley-y.pdf_tex}
    \end{center}
\end{figure}


\subsection{Άξονας X}
\label{subsec:construct:x}

Στην ίδια πλάκα, στερεώνεται το ένα άκρο του γεφυρώματος, ένας νέου οδηγού, ο
οποίος ενώνει αυτή με την απέναντι πλευρά της βάσης, πάνω στο οποίο εκτελείται η
κίνηση στον άξονα X.
Ωστόσο, επιλέγεται ελαφρώς διαφορετική προσέγγιση από αυτήν που χρησιμοποιήθηκε
για τον άξονα Y. Βασικό λόγο αποτελεί η τοποθέτηση του κινητήρα και αυτού του
άξονα στην ίδια πλάκα με αυτήν του κινητήρα του άξονα Y, προκειμένου η καλωδίωση
του να εκτείνεται προς ένα κινητό σημείο και όχι, δυνητικά, όλου του πλάτους της
βάσης. Επίσης, με αυτόν τον τρόπο αξιοποιείται μεγαλύτερο μέρος της επιφάνειας
της πλάκας γεφυρώματος.

Στο σχήμα \ref{fig:construct:x-axis-schem} παρουσιάζεται η υλοποίηση του άξονα
X.
Ο κινητήρας X τοποθετείται κοντά στο κέντρο της πλάκας γεφυρώματος (β) με τη
βάση της τροχαλίας (α) παράλληλα προς την πλάκα και ελαφρώς υψηλότερα της ώστε ο
ιμάντας να ευθυγραμμίζεται με αυλάκωση του γεφυρώματος. Η πλάκα φορτίου του
άξονα X τοποθετείται κάθετα ως προς το δάπεδο και ο ιμάντας προσδένεται στις
σχετικές πλαϊνές της οπές. Στην πλευρά της πλάκας πλησιέστερη της τροχαλίας
(στ), ο ιμάντας προσδένεται απευθείας, ενώ στην άλλη, εφόσον διανύσει το μήκος
του γεφυρώματος και αναστραφεί σε ελεύθερη τροχαλία (η) στο άλλο άκρο του.

\begin{figure}
    \caption{Συστατικά μέρη γεφυρώματος. \label{fig:construct:x-axis-schem}}
    \begin{center}%
    \def\svgwidth{\textwidth}
    \input{img/construct_x-axis.pdf_tex}
    \end{center}
\end{figure}

Στο άκρο της ελεύθερης τροχαλίας, επιλέγεται διαφορετικός τρόπος στερέωσης και
ανύψωσης του οδηγού από τη βάση. Ουσιαστικά, απαλείφεται η πλάκα γεφυρώματος και
χρησιμοποιούνται δύο τροχοί αντί τεσσάρων, οι οποίοι στερεώνονται απευθείας στον
οδηγό. Ο βασικότερος λόγος είναι η αξιοποίηση μεγαλύτερου μήκους του οδηγού,
καθώς μία πλάκα γεφυρώματος, συνολικού μήκους 12.7cm, επιφέρει μεγαλύτερο
περιθώριο (θ) εσωτερικά της βάσης από ότι ένας τροχός. Η επιλογή αυτή
ενδυναμώνεται περαιτέρω από το γεγονός ότι μία πλάκα στο σημείο αυτό θα
χρησιμοποιούταν μόνο για τους τροχούς και την ελεύθερη τροχαλία.

Ωστόσο, παραμένει βασική προϋπόθεση η χρήση κάποιου εξαρτήματος στη θέση της
πλάκας που έχει το ίδιο πάχος, ώστε να διατηρείται οριζόντιος ο οδηγός. Το
μικρότερο εξάρτημα που εντοπίστηκε είναι το πώμα (ζ) το οποίο, τυπικά,
προορίζεται για την κάλυψη των άκρων των οδηγών. Η ορατή έδρα του στο σχήμα, η
οποία διαθέτει μία κοιλότητα, εφάπτεται στον οδηγό. Η άλλη έδρα είναι πλήρως
επίπεδη και σε αυτήν ακουμπά το διαχωριστικό. Επομένως, δεδομένου ότι
χρησιμοποιούνται διαχωριστικά (δ) και επίσωτρα (γ) ίδιου ύψους, ο οδηγός
διατηρείται οριζόντιος.

Η ελεύθερη τροχαλία δημιουργείται με επίσωτρο που παρέχει το σύστημα κατασκευής
με τρόπο αντίστοιχο αυτού της σύνθεσης των τροχών. Συνεπώς, ελεύθερες τροχαλίες
μπορούν να τοποθετηθούν απευθείας σε αυλακώσεις οδηγών ή σε πλάκες γεφυρώματος.
Επίσης, παρέχεται και μία διαφορετική μικρότερη πλάκα με λιγότερες οπές ειδικά
για αυτόν το σκοπό (απεικονιζόμενη στο σχήμα \ref{fig:construct:x-axis-schem}η).

Απευθείας προσάρτηση της ελεύθερης τροχαλίας σε αυλάκωση του οδηγού αποκλείεται
σε αυτήν την περίπτωση, καθώς οι διαθέσιμες αυλακώσεις καθιστούν αδύνατη την
άμεση επικοινωνία της με την κινητήρια τροχαλία. Εφόσον έχει ήδη αποκλειστεί η
χρήση πλάκας γεφυρώματος για την αποφυγή άσκοπης σπατάλης χώρου, επιλέγεται η
προσάρτηση της ελεύθερης τροχαλίας στην ειδική πλάκα και, μέσω αυτής, είτε στην
επάνω αυλάκωση του οδηγού (όπως και στο σχήμα \ref{fig:construct:x-axis-schem}),
είτε, εναλλακτικά, στην κάτω αυλάκωση, αντικαθιστώντας το εξωτερικό πώμα.


\subsection{Άξονας Z}
\label{subsec:construct:z}

\label{sec:construct:z-axis}
Τελικά, στην πλάκα φορτίου του άξονα X στερεώνεται το κατώτερο τμήμα οδηγού για
την κίνηση στον άξονα Z, με το μεγαλύτερο τμήμα του να εκτείνεται πάνω από την
πλάκα. Και σε αυτήν την περίπτωση επιλέγεται η χρήση ιμάντα σε συνδυασμό με
κινητήρια και ελεύθερη τροχαλία. Επειδή ενδιαφέρει η μετακίνηση μόνο των
αισθητήρων που προορίζονται για την παρακολούθηση του υλικού, επιλέγεται η πλάκα
Mini~V, η οποία χαρακτηρίζεται από πολύ μικρό μέγεθος, και πάνω σε αυτήν
προσαρτώνται οι αισθητήρες, όπως φαίνεται αριστερά στο σχήμα
\ref{fig:construct:z-axis}. Η εμφανιζόμενη θήκη αποτελεί μέρος των προμηθευμένων
αισθητήρων υγρασίας και οξύτητας η οποία έχει τροποποιηθεί ώστε να στεγάζεται
ένας ακόμα αισθητήρας, αυτός της θερμοκρασίας (γ), και για τη δημιουργία
ορισμένων πρόσθετων οπών για τους κοχλίες και τις γραμμές σήματος.

\begin{figure}
    \caption{Αναπαράσταση οδηγού άξονα Z. \label{fig:construct:z-axis}}
Το μήκος του οδηγού είναι ενδεικτικό και όχι αντιπροσωπευτικό του πραγματικού.
Σημειώνεται ότι στο σχήμα δεξιά εμφανίζεται η διατομή ορισμένων στοιχείων.
    \begin{center}%
    \def\svgwidth{0.5\textwidth}
    \input{img/construct_z-axis.pdf_tex}
    \end{center}
\end{figure}

%belt path
Δεξιά του ίδιου σχήματος εμφανίζεται η διάταξη των στοιχείων που συνθέτουν τον
οδηγό Z. Όπως γίνεται αντιληπτό και από το σχήμα, οι βίδες και τα προσαρτήματα
για τη στερέωση του οδηγού που βρίσκονται χαμηλά στην οπίσθια αυλάκωσή του (α),
αποτρέπουν τη διέλευση του ιμάντα από αυτήν. Για το λόγο αυτό επιλέγεται οδηγός
VSlot~40 ώστε, εναλλακτικά της αυλάκωσης, να αφιερώνεται η ενδιάμεση σήραγγα ως
δίοδος επιστροφής του ιμάντα.

Όπως φαίνεται και στο σχήμα, η ελεύθερη τροχαλία στερεώνεται λίγο διαφορετικά σε
σχέση με τους τρόπους που έχουν προηγουμένως αναφερθεί, αντικαθιστώντας την
ειδική πλάκα με δύο απλά προσαρτήματα. Ο λόγος είναι η αξιοποίηση περισσότερου
μήκους των αισθητήρων, εφόσον τα απλά προσαρτήματα δεσμεύουν λιγότερο χώρο της
αυλάκωσης στην οποία κινείται και η πλάκα Mini~V με αποτέλεσμα, η τελευταία, να
μετακινείται χαμηλότερα στον άξονα Z.

Δεδομένου ότι η πλάκα φορτίου του άξονα X (β) εκτείνεται κάτω από τον οδηγό της,
πρέπει να εξασφαλίζεται ότι και αυτή καθώς και τα υπόλοιπα μέρη, κινούνται χωρίς
να παρεμποδίζονται από τρίτα αντικείμενα. Αυτό αιτιολογεί τη χρήση αρκετά πλατύ
οδηγού για το πλαίσιο της βάσης, όπως αναφέρεται στην ενότητα
\ref{sec:construct:base}.


\section{Τοποθέτηση κινητήρων}
\label{sec:construct:motors}

Ενώ το σύστημα κατασκευής OpenBuilds παρέχει πολλά εξαρτήματα για την κάλυψη
πληθώρας αναγκών, έχει, ωστόσο, αποβεί αδύνατος ο εντοπισμός κάποιων που να
διευκολύνουν την τοποθέτηση κινητήρων servo. Αντιθέτως, το σύστημα Actobotics
είναι προσανατολισμένο προς τέτοιους κινητήρες. Μολονότι τα δύο συστήματα
παρουσιάζουν χαμηλή συμβατότητα στα εξαρτήματά τους, μπορούν να συνυπάρξουν, έως
ένα βαθμό.

\begin{figure}
    \caption{Κανάλι Actobotics και διάταξη οπών.
        \label{fig:construct:acto-channel-pattern}}
    \begin{center}
        \begin{subfigure}[b]{0.40\textwidth}
            \includegraphics[width=0.95\textwidth]{construct_acto-channel-3in.png}
            \caption{}
        \end{subfigure}
        \begin{subfigure}[b]{0.35\textwidth}
            \includegraphics[width=0.9\textwidth]{construct_acto-pattern.png}
            \caption{}
        \end{subfigure}
    \end{center}

    (βʹ): \fullcite{actobotics:channel-pattern}
\end{figure}

Η λύση εντοπίζεται στις κυκλικά διατεταγμένες οπές των καναλιών -- αντίστοιχων
μονάδων των οδηγών VSlot -- οι οποίες απέχουν 0.77in, περίπου 2cm, (σχήμα
\ref{fig:construct:acto-channel-pattern}β), δηλαδή όσο και οι οπές πλακών και
περικοχλίων του OpenBuilds, με αποτέλεσμα να μπορούν να συνδεθούν στα σημεία
αυτά.
Επομένως, επιλέγεται να χρησιμοποιηθούν εξαρτήματα Actobotics για την πλαισίωση
των κινητήρων και, ως βάση, το κανάλι του εν λόγω συστήματος το οποίο στη
συνέχεια προσδένεται σε κάποιο εξάρτημα του συστήματος OpenBuilds. Στην
περίπτωση του κινητήρα Z, το κανάλι προσδένεται απευθείας σε αυλάκωση του 
αντίστοιχου οδηγού ενώ στην περίπτωση των X και Y, πάνω στην πλάκα γεφυρώματος.

Μία κατηγορία εξαρτημάτων του συστήματος Actobotics ιδιαίτερου ενδιαφέροντος
είναι τα στηρίγματα κινητήρα, τα οποία επιτρέπουν πολλές και σύνθετες διατάξεις
τους. Μολονότι παρέχεται στήριγμα που στερεώνεται απευθείας στις γωνιακές οπές
του καναλιού, επιλέγεται στήριγμα που απαιτεί πρόσθετα γωνιακά προσαρτήματα ώστε
ο κινητήρας να εξωθείται ελαφρώς υψηλότερα από το κανάλι προκειμένου να
αξιοποιείται κατά το μέγιστο δυνατό ο χώρος στο εσωτερικό του καναλιού.
Η ανάγκη αυτή προκύπτει για τον κινητήρα του άξονα X όπου η τροχαλία βρίσκεται
στο εσωτερικό του καναλιού. Για περισσότερη ομοιομορφία μεταξύ των κινητήρων,
επιλέγεται η ίδια διάταξη για όλους, παρότι δεν υφίσταται αντίστοιχη επιτακτική
ανάγκη.

\begin{figure}
    \caption{Διάταξη κινητήρων αξόνων X και Y.\label{fig:construct:xy-servo}}
    \begin{center}%
    \def\svgwidth{0.5\textwidth}
%    \input{img/construct_xy-servo.pdf_tex}
    \includegraphics[width=\textwidth]{construct_xy-servo.png}
    \end{center}
\end{figure}
%
%Επιλέγονται γωνιακά προσαρτήματα για την εξώθηση του στηρίγματος του κινητήρα
%ελαφρώς υψηλότερα από το κανάλι ώστε να αξιοποιείται κατά το μέγιστο δυνατό ο
%χώρος στο εσωτερικό του. Η ανάγκη αυτή προκύπτει για τον κινητήρα του άξονα X
%όπου η τροχαλία βρίσκεται στο εσωτερικό του καναλιού. Για περισσότερη
%ομοιομορφία μεταξύ των κινητήρων, επιλέγεται η ίδια διάταξη για όλους, παρότι
%δεν υφίσταται αντίστοιχη επιτακτική ανάγκη.

Στο σχήμα \ref{fig:construct:xy-servo} παρουσιάζεται η διάταξη των στοιχείων για
τους κινητήρες των αξόνων X και Y.
Στην άτρακτο του κινητήρα προσαρτάται σύνδεσμος και, δια μέσω αυτού, το ένα άκρο
ράβδου περιστροφής. Για τους κινητήρες X και Z, η ράβδος είναι αρκετά μακρυά
ώστε να διέρχεται από ένσφαιρο τριβέα τοποθετημένο σε οπή 0.5in του καναλιού. Ο
επιπρόσθετος τριβέας αυξάνει την σταθερότητα της ράβδου, ιδίως όταν στα άκρα της
εφαρμόζεται ακτινικό φορτίο (μέσω του ιμάντα).

Η διατομή της ράβδου είναι σχήματος D και όχι κυκλική ώστε να δημιουργείται μία
επίπεδη επιφάνεια η οποία εξυπηρετεί την ασφαλέστερη πρόσδεση εξαρτημάτων που
διαθέτουν ένθετο κοχλία σύσφιξης. Τόσο ο σύνδεσμος όσο και η κινητήρια τροχαλία
επωφελούνται από αυτήν την ιδιότητα της ράβδου.


%[εικόνα SPLINE SHAFT-COUPLER !BOLT!]

%OPTOCODER

%TMs

%bracket	προσάρτημα στήριξης	Γ.Χαλκιαδάκης,διπλ.αεροναυπηγός μηχανικός

%set screw	κοχλίας πρόσδεσης	ΤΕΕ
%	κοχλίας αξονικής σύσφιξης	Κ.Μπουζάκης,Καθηγητής,Σχολή Μηχανολόγων Μηχανικών

%radial load	ακτινικό φορτίου	JAR 23, Greek CAA;
%		ΒΑΣ.ΦΙΛΟΠΟΥΛΟΣ,Χημικός-Μηχανικός

%punched tape	διάτρητη χαρτοταινία	ΕΠΥ

%machining	μηχανουργική κατεργασία	ΤΕΕ

%structural system	δομικό σύστημα	Dec. 96/582/EC OJ L 254/96 p.62

%plastic deformation	πλαστική παραμόρφωση
%		ΕΛΕΤΟ/Ειδική Ομάδα Ορολογίας Μεταλλουργίας-Μεταλλογνωσίας, Έγκριση ΓΕΣΥ

%cantilevered	πρόβολος δοκός	Κ.Μπουζάκης, καθηγ., σχολή Μηχανολόγων Μηχανικών Α.Π.Θ.

%moving table	κινούμενη τράπεζα	ΙCG-ΕΛΛΗΝΙΚΟΣ ΥΑΛΟΥΡΓΙΚΟΣ ΣΥΝΔΕΣΜΟΣ

%idler pulley	ελεύθερη τροχαλία // τροχαλία έντασης

%pulley	κινητήρια τροχαλία

%τροχαλία με αυλακωτή στεφάνη
