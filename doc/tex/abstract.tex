\thispagestyle{empty}

\begin{center}{\bfseries Περίληψη}\end{center}

\noindent
Η εργασία ασχολείται με την υλοποίηση μίας συσκευής που αυτοματοποιεί την
παρακολούθηση συνθηκών που επικρατούν στο υλικό ανοικτού δοχείου το οποίο αυτή
περιβάλλει. Η πρόσβαση στη συσκευή πραγματοποιείται απομακρυσμένα με χρήση
πρωτοκόλλου HTTP.
%
Η υλοποίηση στηρίζεται στη χρήση μίας πλακέτας \te{Arduino Uno rev3} και ενός
μικροελεγκτή Atmel AVR ATmega328 που αυτή περιέχει. Ο προγραμματισμός του
μικροελεγκτή γίνεται δια μέσω λογισμικού \te{Boot loader} και διασύνδεσης USB.
Η ανάπτυξη του λογισμικού υποβοηθείται μόνο από την AVR GNU συλλογή
μεταγλωττιστών avr-gcc και τη βιβλιοθήκη προγραμματισμού AVR Libc. Κρίσιμες
δυνατότητες που ο συγκεκριμένος μικροελεγκτής στερείται, όπως ρολόι πραγματικού
χρόνου, δικτυακή διασύνδεση και επιπρόσθετο αποθηκευτικό χώρο, παρέχονται από
εξωτερικά ολοκληρωμένα κυκλώματα.
%
Η συσκευή ρυθμίζεται σε σχέση με παραμέτρους βοηθητικών λειτουργιών της καθώς
και της διαδικασίας παρακολούθησης. Όλες οι ρυθμίσεις διατηρούνται μεταξύ
διακοπών τροφοδοσίας και είναι δυνατό να επαναφερθούν στις εργοστασιακές
της ρυθμίσεις με μηχανικό τρόπο.
%
Για τη διασύνδεσή της με το χρήστη, υλοποιείται λογισμικό διακομιστή HTTP μέσω
του οποίου επιτυγχάνεται η ανάκτηση πόρων, οι οποίοι, σε συνδυασμό με
παρεχόμενες μεθόδους του πρωτοκόλλου HTTP, επιτρέπουν πρόσβαση σε αυτήν.
Η μορφή των αναπαραστάσεων των πόρων γίνεται σε JSON και προορίζεται για
αξιοποίηση από εξωτερικές εφαρμογές. Επιπλέον, υποστηρίζονται πόροι σε HTML,
CSS, JavaScript και PNG για χρήση της συσκευής μέσω λογισμικού πλοήγησης.
%
Το κεντρικό σημείο ενδιαφέροντος της εργασίας έγκειται στην υλοποίηση ενός
ολοκληρωμένου συστήματος που εμπλέκει πληθώρα τεχνολογικών πεδίων.

\clearpage
\thispagestyle{empty}

\begin{center}{\bfseries Abstract}\end{center}

\begin{english}
\noindent
The project's aims is a device that automates monitoring the conditions of the
material within a container. Access to the device is provided remotely via the
HTTP protocol.
%
The core of the implementation is an Arduino Uno rev3 board and an Atmel AVR
ATmega328 microcontroller contained therein. The microcontroller is programmed
via Boot loader software on a USB interface. The development of the firmware is
based solely on the GNU AVR toolchain and AVR Libc library. Important
functionality not inherently present on the microcontroller, such as real-time
clock, network interfacing and additional storage, is provided by external
integrated circuits.
%
The device may be configured as far as monitoring parameters are concerned
as well as supplementary functions it provides. A mechanism is provided so the
configuration is unaffected from power outage, while allowing to manually reset
it to its default factory settings, if needed.
%
The user interface is based on a rudimentary HTTP server; HTTP resources in
combination with HTTP methods provide the fundamental mechanism for remote
access. The resources are represented in JSON format, mostly intended for
external software systems. A few additional resources are provided in HTML, CSS,
JavaScript and PNG intended to be viewed in a web browser.
%
The primary reason for undertaking such a project lies in the desire to
implement a complete system that incorporates various technological fields.
\end{english}

\clearpage
