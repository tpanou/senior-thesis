\chapter{Συμπεράσματα}

Η εργασία παρείχε την ευκαιρία ενασχόλησης με την ανάπτυξης ενός ολοκληρωμένου,
μολονότι απλοϊκού, συστήματος αυτοματισμού κάνοντας χρήση σύγχρονων τεχνολογιών,
από την αρχή μέχρι το τέλος του.

Ανέδειξε την αξία της ανεξάρτητης μελέτης στην αναζήτηση και αξιολόγηση πηγών
και την ανάλυση τεχνικών εγγράφων για την κάλυψη του απαραίτητου γνωστικού
υποβάθρου. Έδωσε την ευκαιρία πειραματισμού με γνωστικά πεδία πέραν των ορίων
της Πληροφορικής και επέτρεψε τη συλλογή πλούτου εμπειρίας και γνώσεων σε
ποικίλους τομείς. Μέσω αυτής επιβεβαιώθηκε, για άλλη μία φορά, η αξία των
εφοδίων που παρέχει το Πρόγραμμα Σπουδών και αποτελεί έναυσμα για την διεύρυνσή
τους.

Το αποτέλεσμα της εργασίας πλησιάζει ικανοποιητικά τον προσδοκώμενο στόχο.
Παρόλα αυτά, στο υπόλοιπο του κεφαλαίου αναφέρονται ορισμένες ατέλειες που έχουν
παρατηρηθεί, εναλλακτικές προσεγγίσεις ορισμένων θεμάτων που θα μπορούσαν να
βελτιώσουν το συνολικό αποτέλεσμα καθώς και τρόπους για μία πιθανή επέκταση των
δυνατοτήτων του.


\section{Απρόσμενες συμπεριφορές}

Για λόγους πληρότητας, αναφέρεται ότι έχουν παρατηρηθεί ορισμένες απρόσμενες
συμπεριφορές. Όπως γίνεται αντιληπτό από την περιγραφή τους παρακάτω, και οι
τρεις περιπτώσεις παρατηρημένες περιπτώσεις ενδέχεται να είναι προϊόν παραγόντων
πέραν των ορίων του λογισμικού που οφείλονται και σε φαινόμενα ηλεκτρονικής
ακόμα και μηχανικής φύσεως.


\subsection*{Αδυναμία επαναφοράς από όριο Y\protect\tsub{max}}

Μία εξ αυτών σχετίζεται με την κίνηση στον άξονα Y -- μία εκ των διευθύνσεων
κίνησης της κεφαλής μετρητών -- και, για την ακρίβεια, με την ακινητοποίηση του
κινητήρα ως αποτέλεσμα πρόσκρουσης στο τέλος της διαδρομής. Τυπικά, ο
μικροελεγκτής αλλάζει το αποστελλόμενο σήμα ώστε ο κινητήρας να οπισθοδρομήσει
από το άκρο μέχρι να απελευθερωθεί ο ανασταλτικός διακόπτης. Κατόπιν τίθεται σε
πορεία που οδηγεί την κεφαλή πίσω στη θέση 0 του άξονα.

Ωστόσο, αυτό που συμβαίνει είναι ότι τη στιγμή που ενεργοποιείται ο διακόπτης,
ο κινητήρας συνεχίζει να περιστρέφεται με την ίδια φορά με ασθενέστερο, βέβαια,
ρυθμό. Αυτή η συμπεριφορά παρατηρείται κατά την εξώθηση από το μέγιστη θέση και
μόνο για τον άξονα Y.


\subsection*{Πρόωρη ολοκλήρωση μετατόπισης}

Μία δεύτερη ανεξήγητη, μέχρι πρότινος, συμπεριφορά σχετίζεται, εν μέρει, με το
υποσύστημα κίνησης και το υποσύστημα διακομιστή HTTP. Εφόσον το πρώτο έχει
τεθεί για τη μετατόπιση της κεφαλής και ενώ αυτή βρίσκεται εν κινήσει, εάν ο
διακομιστής αρχίσει την επεξεργασία κάποιου εισερχόμενου αιτήματος,
\emph{ενδέχεται} η κίνηση της κεφαλής να ολοκληρωθεί πρόωρα.

Από προσπάθειες απασφαλμάτωσης, ο Χρονομετρητής\slash{}Μετρητής που καταμετρά
τα βήματα των κινητήρων παρουσιάζεται ότι έχει όντως καταμετρήσει το πλήθος
βημάτων που του είχαν δηλωθεί. Ωστόσο, εάν τα δεδομένα που παραλαμβάνονται από
το \te{Socket} του διακομιστή, απορριφθούν χωρίς να καταναλωθούν (δηλαδή, χωρίς
να εισέλθουν) μέσα στο μικροελεγκτή, τότε η μετατόπιση της κεφαλής ολοκληρώνεται
πάντα σύμφωνα με το αναμενόμενο.

Σημειώνεται ότι τα εισερχόμενα δεδομένα των αιτημάτων παραμένουν σε προσωρινή
μνήμη (\te{buffer}) του εξωτερικού ολοκληρωμένου δικτύωσης και υπόκεινται
επεξεργασία καθώς αυτά μεταφέρονται στο μικροελεγκτή μέσω διαύλου SPI. Εικάζεται
ότι οι γραμμές του διαύλου ίσως προκαλούν παρεμβολές στις γραμμές από όπου
διέρχονται τα βήματα των κωδικοποιητών, εισάγοντας κίβδηλους παλμούς. Η ιδέα ότι
μπορεί να υφίσταται κάτι τέτοιο έχει δοθεί από οδηγίες της
\textcite[7]{maxim:xtal} που προειδοποιούν στην περίπτωση όπου ένα
ολοκληρωμένο ρολόι πραγματικού χρόνου (RTC) εκτελείται πιο γρήγορα του
κανονικού, ενδέχεται να οφείλεται από τη σύζευξη θορύβου από γειτονικά σήματα
στους ακροδέκτες με τους οποίους το RTC συνδέεται με τον εξωτερικό ταλαντωτή.


\subsection*{Μη ανταπόκριση ολοκληρωμένου δικτύωσης}

Μία τρίτη και τελευταία περίπτωση αφορά το ολοκληρωμένο δικτύωσης το οποίο
παρατηρήθηκε ότι σε, φαινομενικά, τυχαίες στιγμές, η λυχνία δεδομένων
(\te{trafic}) που με αυτό συνδέεται, πάλλεται εξακολουθητικά δηλώνοντας ότι
είναι μονίμως απασχολημένο, αποτρέποντας τη συσκευή από την εξυπηρέτηση άλλων
αιτημάτων.


\section{Βελτιώσεις}
\label{sec:improvements}

\subsection*{Υποσύστημα κίνησης}

Αρκετά είναι τα σημεία που σχετίζονται με την κίνηση της κεφαλής που θα
μπορούσαν να δεχθούν βελτιώσεις. Μία σχετίζεται με τους ίδιους τους κινητήρες·
επί του παρόντος χρησιμοποιούνται κινητήρες \te{servo} οι οποίοι επελέγησαν για
ενδεχόμενη περισσότερη ευκολία διασύνδεσης με το μικροελεγκτή. Ωστόσο, εφόσον
ενδιαφέρει η ακρίβεια της κίνησης της κεφαλής έναντι της ταχύτητας, ίσως
βηματικοί κινητήρες (\te{stepper}) να αποβούν καταλληλότεροι.

Η ανάγκη γίνεται ιδιαίτερα αισθητή κατά την μετατόπιση της κεφαλής στο επίπεδο
X-Y για το διάστημα που ενεργοποιούνται και οι δύο κινητήρες. Η βασική παραδοχή
της υλοποίησης βασίζεται στην υπόθεση ότι σε δεδομένο χρονικό διάστημα,
επιτυγχάνεται η ίδια (ή με μικρή απόκλιση) μετατόπιση και στους δύο άξονες (X
και Y). Αυτό γίνεται για την αντιμετώπιση του περιορισμένου αριθμού κυκλωμάτων
Χρονομετρητών\slash{}Απαριθμητών όπου καταμετρούνται τα βήματα μόνο του ενός εκ
των δύο κωδικοποιητών κίνησης. 

Πιθανές λύσεις είναι αρκετές, μία εξ αυτών είναι η χρήση κάποιου πιο εξελιγμένου
μικροελεγκτή ή ενός αφιερωμένου για τον έλεγχο των κινητήρων ελεγκτή (\te{motor
controller}). Μία πιο απλή λύση υπό τις παρούσες συνθήκες θα ήταν η υποστήριξη
μετατόπισης μόνο σε έναν άξονα τη φορά. Τέλος, θα μπορούσε να βελτιωθεί η
ακρίβεια (ή διακριτική ικανότητα) των κωδικοποιητών κίνησης όπως, για
παράδειγμα, με τη χρήση έτοιμων αντίστοιχων λύσεων.


\subsection*{Εργασία μετρήσεων}

Άλλο σημείο που θα μπορούσε να βελτιωθεί είναι η (αυτόματη) εκκίνηση νέων
εργασιών (δηλαδή, κύκλων μετρήσεων).
Εάν η συσκευή παραμείνει απενεργοποιημένη για παρατεταμένο χρόνο, όταν τελικά
συνδεθεί με την τροφοδοσία ενδέχεται να μην εκκινηθεί νέα μέτρηση αμέσως.
Κρίθηκε ότι η συγκεκριμένη συμπεριφορά είναι αποδεκτή για τις ανάγκες της
τρέχουσας υλοποίησης. Ωστόσο, θα μπορούσε να τροποποιηθεί ώστε παρακολουθείται
ολόκληρη η ημερομηνία αντί απλώς μόνο των ωρών, στις οποίες και οφείλεται αυτή
η συμπεριφορά.

Επιπλέον, θα μπορούσε να υλοποιηθεί ένας πιο εκλεπτυσμένος αλγόριθμος επιλογής
θέσεων για την πραγματοποίηση μετρήσεων ώστε, για την επιλογή των θέσεων, να
λαμβάνεται υπόψη οι θέσεις των πρόσφατα πραγματοποιηθέντων μετρήσεων καθώς
και αυτών που είχαν σημειώσει κρίσιμες (οριακές) μετρήσεις.


\subsection*{Θέματα ισχύος}

Επιπλέον, η μείωση της κατανάλωσης ισχύος εφαρμόζεται μόνο στο μικροελεγκτή και
το υποσύστημα κίνησης (τόσο για τους κωδικοποιητές όσο και για τους κινητήρες),
με διαφόρους τρόπους· ο μικροελεγκτής περνά τον περισσότερο χρόνο του σε
κατάσταση \te{power-down}, ενώ οι αισθητήρες των κωδικοποιητών (κυρίως του
πομπού) διαρρέονται από ρεύμα μόνο κατά τη λειτουργία των κινητήρων.

Θα μπορούσε να εφαρμοστεί και στα υπόλοιπα ολοκληρωμένα κάποια αντίστοιχη
τακτική. Για παράδειγμα, η εξωτερική μνήμη \te{Flash} διαθέτει λειτουργία
χαμηλής κατανάλωσης (\te{Deep power-down mode}) στην οποία μπορεί να τεθεί, κατά
την οποία η ένταση του ρεύματος που απαιτεί, μειώνεται σε, κατά μέγιστο, 1μA από
10μA που σημειώνεται σε κατάσταση ετοιμότητας \parencite[2,16]{25lc1024}.

% Χρησιμοποιούνται δύο καλώδια τροφοδοσίας· ένα για τον μικροελεγκτή και τα
% συναφή και ένα για τους κινητήρες.

\subsection*{Υποσύστημα κίνησης}

Τέλος, ένα άλλο κομμάτι το οποίο θα μπορούσε να δεχθεί βελτιώσεις είναι το
κομμάτι του διακομιστή HTTP. Επί του παρόντος, παρότι έχει δοθεί βαρύτητα στην
ευελιξία του διακομιστή όσον αφορά τον προγραμματισμό των διατιθέμενων πόρων και
τη σύνταξη των πεδίων κεφαλίδας των αποκρίσεων, ο υποκείμενος αναλυτής πεδίων
κεφαλίδας καθώς και ο αναλυτής\slash{}συντάκτης αναπαραστάσεων JSON έχουν αρκετά
σημεία που χρήζουν αναθεώρησης.


\section{Εξέλιξη}

Πέραν των πιθανών βελτιώσεων, η υλοποίηση θα μπορούσε να επεκταθεί ώστε να
αυτοματοποιεί μεγαλύτερο μέρος της εργασίας. Μία μορφή θα ήταν η υποστήριξη
περισσότερων αισθητήρων· παρέχεται ήδη καταμέτρηση της θερμοκρασίας, ωστόσο θα
μπορούσε να συμπεριληφθούν η υγρασία και, ενδεχομένως, η οξύτητα του υλικού που,
μαζί με τη θερμοκρασία αποτελούν τις βασικές παραμέτρους για τη διατήρηση ενός
υγιούς μέσου.

Ένα επόμενο στάδιο, θα ήταν η προσθήκη δυνατοτήτων επενέργειας στο υλικό ώστε,
εκτός από παρακολούθηση, η συσκευή να τροποποιεί τις συνθήκες του υλικού. Για
παράδειγμα, η ενσωμάτωση ενός εξαρτήματος στη βάση της κεφαλής το οποίο
χρησιμοποιείται για την ύγρανση του υλικού σε περιοχές όπου καταγράφηκε υγρασία
κάτω από τα καθορισμένα, από το χρήστη, όρια.

Επιπλέον, θα μπορούσε η ίδια η συσκευή να αναγγέλλει τις καταχωρημένες μετρήσεις
της σε τρίτα συστήματα ή και να ειδοποιεί στην περίπτωση που προκύπτει κάποιο
κρίσιμο συμβάν (\te{alarm}), όπως, για παράδειγμα, επικίνδυνα υψηλή θερμοκρασία
ή πολλαπλές αστοχίες υλικού (του υποσυστήματος κινητήρων).
%Για περαιτέρω
%η συσκευή θα μπορούσε να υλοποιεί πρωτόκολλο DCHP ώστε να αποδίδεται αυτόματα
%διεύθυνση δικτύου και να μην απαιτείται επαναφορά στις εργοστασιακές ρυθμίσεις
%της όποτε !!!

Στο κομμάτι της διεπαφής HTTP της συσκευής, θα μπορούσε να εφαρμοστεί διαδικασία
διαπίστευσης χρήστη για την προφύλαξη ορισμένων ευαίσθητων λειτουργιών που
παρέχει όπως η τροποποίηση των ρυθμίσεών της (για παράδειγμα, διεύθυνση IP). Το
συγκεκριμένο παραλήφθηκε, εσκεμμένα, από την τρέχουσα υλοποίηση εξαιτίας των
χρονικών περιθωρίων.

Επιπλέον, θα μπορούσε να υποστηριχθούν πιο εύχρηστες ή διαδραστικές μορφές
παρουσίασης των μετρήσεων. Στην τρέχουσα κατάσταση, οι μετρήσεις εμφανίζονται
σελιδοποιημένες σε πίνακα μεγέθους που επιλέγει ο χρήστης, χωρίς, ωστόσο, να
δίνεται η δυνατότητα ταξινόμηση ή να παρέχεται κάποια γραφική απεικόνιση τους.
Βέβαια, κάτι τέτοιο θα μπορούσε να υλοποιηθεί από τρίτο σύστημα το οποίο
ενδεχομένως συγκεντρώνει στατιστικά στοιχεία και από άλλες παρόμοιες συσκευές.
