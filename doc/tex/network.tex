
%W5100

Η χρήση του ολοκληρωμένου επιλέγεται να γίνει δια μέσω της πλακέτας WIZ811MJ της
ίδιας εταιρείας. Οι λόγοι είναι ότι επιλύει τη διασύνδεση του ολοκληρωμένου με
συνδετήρα RJ-45 (απαραίτητος για την ανταλλαγή δεδομένων μεταξύ W5100 και
δικτύου), ταλαντωτή και διάφορα άλλα ηλεκτρονικά στοιχεία. Από τους 80,
συνολικά, ακροδέκτες του W5100, το WIZ811MJ διαθέτει μόλις 40, οι οποίοι είναι
αυτοί που προορίζονται για τη διασύνδεση με μικροελεγκτή. Όλοι οι υπόλοιποι
συμμετέχουν σε εσωτερικές συνδέσεις της πλακέτας.

Ένα παράδειγμα σχετικά με το τελευταίο, το W5100 διαθέτει έναν ακροδέκτη για το
χειρισμό του ολοκληρωμένου ως Slave σε δίαυλο SPI, το \nbar{SCS}
(\textenglish{SPI Chip-Select}) \parencite[9]{wiz11:w5100}. Ωστόσο, διαθέτει και
έναν επιπρόσθετο ακροδέκτη, τον SEN (\textenglish{SPI Enable}), για την
ενεργοποίηση των υποσυστημάτων του ολοκληρωμένου για λειτουργία σε SPI
\parencite[8]{wiz11:w5100}. Το WIZ811MJ παρέχει ακροδέκτη μόνο για το \nbar{SCS}
ενώ διαθέτει κατάλληλη διάταξη ώστε η αλλαγή της τιμής του να προκαλεί το
αναμενόμενο σήμα στο SEN, αυτομάτως \parencite[7]{wiz13:811mj}.

Επιπλέον σημαντικό χαρακτηριστικό είναι ότι η πλακέτα παρέχει ακροδέκτες
συμβατούς με πρωτότυπες κάρτες (\textenglish{breadboard}) που χρησιμοποιεί η
υλοποίηση, εν αντιθέσει με το W5100 το οποίο είναι SMD
(\textenglish{Surface-Mount Device}) και απαιτεί συγκόλληση
\parencite[6,12]{wiz13:811mj}.

%Αποτελεί μία έτοιμη λύση για την προσθήκη δικτυακών δυνατοτήτων στην υλοποίηση.
%W5100 + MAG-JACK


\subsection{Επικοινωνία με μικροελεγκτή}

Το W5100 υποστηρίζει τρεις τρόπους για την επικοινωνία με το μικροελεγκτή· άμεση
ή έμμεση προσπέλαση ή, μέσω πρωτοκόλλου SPI \parencite[59]{wiz11:w5100}. Οι δύο
πρώτες μέθοδοι χρησιμοποιούν διαύλους διεύθυνσης και δεδομένων απαιτώντας πολλά
σημεία σύνδεσης με το μικροελεγκτή· για την ακρίβεια, 3 γραμμές ελέγχου
(\textenglish{Chip-Select, Read, Write}), 8 γραμμές για το δίαυλο δεδομένων,
και, 15 ή 2 γραμμές διεύθυνσης για άμεση ή έμμεση προσπέλαση, αντίστοιχα. Στην
περίπτωση του SPI απαιτούνται πολύ λιγότερες (μόλις 4) και αυτός είναι ο λόγος
που προτιμάται για τη διασύνδεση του με την MCU, δεδομένου του περιορισμένου
αριθμού ακροδεκτών της. Σαφώς, το μειονέκτημα χρήσης SPI -- ενός σειριακού
πρωτοκόλλου επικοινωνίας -- είναι ότι επιτυγχάνεται πολύ μικρότερος ρυθμός
ανταλλαγής δεδομένων από ότι στην περίπτωση των άλλων δύο, κάτι που, τελικά,
επηρεάζει το χρόνο απόκρισης στα εισερχόμενα αιτήματα. Ωστόσο, κρίνεται ότι για
τις ανάγκες της υλοποίησης, αυτός ο περιορισμός είναι αμελητέος σε σχέση με την
εξοικονόμηση ακροδεκτών που επιφέρει η χρήση SPI.

