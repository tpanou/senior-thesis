\chapter{Εισαγωγή}

Η εργασία αναλαμβάνει τη μελέτη, το σχεδιασμό και την υλοποίηση ενός συστήματος
αυτοματισμού με δυνατότητες δικτύωσης.

Το αντικείμενο που επιλέγεται να υποστηριχθεί αντλεί έμπνευση από συστήματα
κομποστοποίησης. Η κομποστοποίηση είναι μία διαδικασία που σκοπό έχει τη
μετατροπή αγροτικών και οικιακών υπολειμμάτων σε πλούσια οργανική μάζα.
Μέσω αυτής, σημειώνονται αρκετά οφέλη για το περιβάλλον και το κοινωνικό σύνολο,
όπως, μείωση όγκου απορριμάτων, εμπλουτισμός εδάφους, ανάληψη ευθύνης απέναντι
στα υποπροϊόντα της ανθρώπινης δραστηριότητας.

Στο πλαίσιο της υλοποίησης επιλέγεται η παρακολούθηση των συνθηκών ενός
υποθετικού δοχείου στο οποίο εναποτίθενται υλικά τα οποία προορίζονται για
αποσύνθεση μέσω ψυχρής κομποστοποίησης που, για την ακρίβεια, κάνει χρήση
γαιοσκώληκα. Σε ένα τέτοιο σύστημα, πρέπει να παρακολουθούνται, κυρίως, η
θερμοκρασία, η υγρασία και η οξύτητα.

Ωστόσο, από τις τρεις αυτές βασικές μεταβλητές, πραγματοποιείται ενδεικτική
υλοποίηση παρακολούθησης μόνο της θερμοκρασίας. Προς επίτευξη αυτού, η συσκευή
διαθέτει ένα ειδικό εξάρτημα που μπορεί να διανύει την επιφάνεια του δοχείου
και να εισχωρεί κατακόρυφα σε αυτό για τη λήψη μετρήσεων.

Τυπικά, η συσκευή ξοδεύει το μεγαλύτερο μέρος της ημέρας σε κατάσταση χαμηλής
κατανάλωσης ισχύος, παρότι συνδεδεμένη στην κεντρική τροφοδοσία. Από αυτήν την
κατάσταση, αφυπνίζεται αυτόματα για την πραγματοποίηση μετρήσεων του υλικού ή
για την απόκριση στις προτροπές του χρήστη.

Σε σχέση με τις αυτόματες μετρήσεις, η συσκευή ρυθμίζεται για τις διαστάσεις του
δοχείου που της προσαρτάται, για το χρονικό διάστημα που είναι επιθυμητό να
παρεμβάλλεται μεταξύ μετρήσεων καθώς και για το πλήθος μετρήσεων που
πραγματοποιούνται κάθε φορά που αφυπνίζεται για το σκοπό αυτό.
Οι θέσεις στις οποίες πραγματοποιούνται μετρήσεις επιλέγονται τυχαία μέσα στο
χώρο του δοχείου, ώστε να παρέχεται μία εικόνα για τη συνολική κατάσταση του
υλικού. Οι 90 πλέον πρόσφατες πραγματοποιηθείσες μετρήσεις αποθηκεύονται σε
μνήμη της συσκευής και φέρουν πληροφορία της ημερομηνίας\slash{}ώρας, της θέσης
και της θερμοκρασίας σε αυτήν.

Η τρέχουσα ημερομηνία\slash{}ώρα καθορίζεται από το χρήστη μία φορά και τηρείται
από κύκλωμά της. Για την ακρίβεια, όλες οι ρυθμίσεις της συσκευής διατηρούνται,
ακόμα και εάν αυτή αποσυνδεθεί από την τροφοδοσία· χαρακτηριστικό ιδιαίτερα
χρήσιμο κατά των διακοπών παροχής ηλεκτρικής ενέργειας.

Μέσα από το λογισμικό περιήγησης της προτίμησής του, δεδομένου ότι αυτό
υποστηρίζει στοιχειωδώς HTML5, CSS3 και Javascript (ΙΕ8/2006 και μετά), από
οποιασδήποτε μορφής υπολογιστή που έχει πρόσβαση στο δίκτυο της συσκευής, ο
χρήστης μπορεί να ενημερώνεται για τις τελευταίες μετρήσεις, να εκτελεί κατά
παραγγελία μετρήσεις στις θέσεις που επιθυμεί και, σαφώς, να τροποποιεί τις
ρυθμίσεις της συσκευής.

Παρότι η σύνδεση του χρήστη με τη συσκευή γίνεται από μία προκαθορισμένη
διεύθυνση, επιτρέπεται η αλλαγή αυτής καθώς και των υπόλοιπων ρυθμίσεων δικτύου,
ώστε να διευκολύνεται η πρόσβαση σε αυτήν μέσα από το τοπικό δίκτυο του εκάστοτε
χρήστη. Ωστόσο, σε περίπτωση που οι ρυθμίσεις αυτές απολεσθούν (κυρίως της
διεύθυνσης IP) με αποτέλεσμα να μην δύναται η πρόσβαση στη διεπαφή της συσκευής,
παρέχεται η δυνατότητα επαναφοράς της συσκευής στις εργοστασιακές της ρυθμίσεις.
Αυτό επιτυγχάνεται με την προσωρινή απομάκρυνση μίας μικρής \te{coin cell}
μπαταρίας που διαθέτει κατά το διάστημα που βρίσκεται απενεργοποιημένη (καλώδιο
τροφοδοσίας αποσυνδεδεμένο).

Ο πιο προχωρημένος χρήστης έχει στη διάθεσή του ένα εύχρηστο προγραμματιστικό
API, το οποίο του επιτρέπει να χειρίζεται τη συσκευή μέσω δικών του εφαρμογών.
Όλη η λειτουργικότητα που παρέχεται μέσω της διεπαφής που προορίζεται για
λογισμικό πλοήγησης, παρέχεται και μέσω του API -- για την ακρίβεια, η διεπαφή
αποτελεί μία τέτοια εφαρμογή που κάνει χρήση αυτού του API.

Το API χρησιμοποιεί την μορφή JSON για την αναπαράσταση των πόρων και
υλοποιείται με την ανταλλαγή μηνυμάτων HTTP και καθιερωμένων μεθόδων (GET, PUT,
POST), ενώ διαγνωστικά μηνύματα, όπως «Το αίτημα μετακίνησης της κεφαλής έγινε
αποδεκτό· αναμενόμενος χρόνος ολοκλήρωσης 7s», επιστρέφονται με τη μορφή
κατάλληλων κωδικών κατάστασης και πεδίων επικεφαλίδας του πρωτοκόλλου HTTP (στο
παράδειγμα, 202 και «Retry-After», αντίστοιχα). Οδηγίες χρήσης παρέχονται μέσα
από τη διεπαφή της συσκευής.
