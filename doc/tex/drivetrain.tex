
\section{Υποσύστημα κίνησης}


%
% Κινητήρες
%

\subsection{Παραγωγή σήματος\slash κίνησης}

Σύμφωνα με εγχειρίδιο χρήσης της \textcite{hitec02}, όλοι οι κινητήρες της
λειτουργούν σε τάση 4.8--6V δεχόμενοι τετραγωνικό σήμα ελέγχου με διαμόρφωση
PWM (\textenglish{Pulse-Width Modulation}) 3--5V σε συχνότητα ανανέωσης 50Hz.
Η διάρκεια των παλμών κυμαίνεται μεταξύ 0.9ms και 2.1ms, με παλμούς των 1.5ms να
χρησιμοποιούνται για την ακινητοποίησή τους \parencite{hitec02}.

Οι παραγωγή των παλμών μπορεί να πραγματοποιηθεί είτε με λογισμικό είτε από
κάποιο διαθέσιμο κύκλωμα Χρονομετρητή\slash Απαριθμητή (\textenglish{Timer\slash
Counter}) του μικροελεγκτή. Το κύριο πλεονέκτημα του λογισμικού είναι η
ελευθερία επιλογής οποιουδήποτε ακροδέκτη για την έξοδο του σήματος καθώς και η
δυνατότητα παραγωγής πολλαπλών τέτοιων σημάτων που περιορίζονται από το συνολικό
πλήθος ακροδεκτών ή τη συχνότητα του ρολογιού συστήματος (όποιο είναι
μικρότερο).

Αντιθέτως, στην περίπτωση χρήσης κυκλωμάτων, το παραγόμενο σήμα διοχετεύεται
στους συγκεκριμένους ακροδέκτες με τους οποίους είναι το καθένα εσωτερικά
συνδεδεμένο και, σαφώς, το μέγιστο πλήθος παράλληλων σημάτων περιορίζεται από το
συνολικό αριθμό τέτοιων κυκλωμάτων. Επίσης, κάθε Χρονομετρητής\slash Απαριθμητής
υπόκειται σε πρόσθετους περιορισμούς, όπως το εύρος τιμών που υποστηρίζει ο
καθένας (για παράδειγμα, ανάλυση 8- ή 16-bit).

Ωστόσο, στη χρήση κυκλώματος Χρονομετρητή\slash Απαριθμητή συγκαταλέγονται και
ορισμένα σημαντικά πλεονεκτήματα.
Καθότι το σήμα παράγεται από ξεχωριστό κύκλωμα, η CPU του μικροελεγκτή είναι
διαθέσιμη για την εκτέλεση οποιασδήποτε άλλης εργασίας (για παράδειγμα, κάποιας
ρουτίνας εξυπηρέτησης διακοπής) παράλληλα με την παραγωγή του σήματος, χωρίς να
απαιτείται πολύπλεξη εργασιών (ώστε να αποδίδονται κβάντα τόσο στη γεννήτρια
σήματος όσο και σε κάποια άλλη εργασία). Ένα δεύτερο, λιγότερο σημαντικό για την
προκειμένη υλοποίηση, πλεονέκτημα είναι η δυνατότητα επίτευξης υψηλότερων
συχνοτήτων με πολύ λιγότερο θόρυβο (\textenglish{jitter}) και μεγαλύτερη
ακρίβεια.

Τελικά, για την παραγωγή σήματος PWM των κινητήρων επιλέγεται η χρήση κυκλώματος
Χρονομετρητή\slash Απαριθμητή καθώς, επιπροσθέτως των ανωτέρω πλεονεκτημάτων,
παρουσιάζει υψηλότερο ενδιαφέρον, από εκπαιδευτικής πλευράς.

\subsubsection{Γεννήτρια PWM}

Ο μικροελεγκτής ATmega328P διαθέτει δύο κυκλώματα Χρονομετρητή\slash Απαριθμητή
με δυνατότητα παραγωγής σήματος PWM. Ο πρώτος (Timer\slash Counter0) είναι 8-bit
και ο δεύτερος (Timer\slash Counter1), 16-bit. Για την επιλογή κάποιου, κρίνεται
σκόπιμη η περαιτέρω ανάλυση της λειτουργίας της γεννήτριας PWM καθώς και των
απαιτήσεων για το σήμα ελέγχου των κινητήρων.

Όσον αφορά τη γεννήτρια PWM, ο Χρονομετρητής\slash Απαριθμητής διαθέτει έναν
καταχωρητή μετρητή, τον TCNTn, ο οποίος σταδιακά αυξάνεται. Επίσης, διαθέτει
δύο ακόμα καταχωρητές, τους OCRnA και OCRnB, οι οποίοι αντιστοιχίζονται με
ακροδέκτες του μικροελεγκτή, αναφερόμενοι ως OCnA και OCnB. Μόλις η τιμή του
TCNTn γίνει ίση με την τιμή που περιέχεται είτε στον OCRnA είτε στον OCRnB,
είναι δυνατή η αλλαγή της εξόδου του αντίστοιχου ακροδέκτη. Ο TCNTn συνεχίζει
την ανοδική του πορεία έως ότου φτάσει την τιμή TOP. Από το σημείο αυτό και
ανάλογα με τη λειτουργία που έχει επιλεγεί, ο TCNTn είτε επανέρχεται ακαριαία
στην τιμή 0 και αρχίζει εκ νέου τον επόμενο κύκλο, είτε φθίνει σταδιακά μέχρι το
0, πάντα εναλλάσσοντας την τιμή των OCnA και OCnB όποτε εξισώνεται με τους OCRnA
και OCRnB, αντιστοίχως %\parencite[]{atmel13}.
Γίνεται αντιληπτό ότι η συχνότητα με την οποία αυξάνεται ο TCNTn καθώς και η
τιμή TOP επηρεάζουν τη συχνότητα του παραγόμενου σήματος. Παράλληλα, η τιμή των
OCRnA και OCRnB επηρεάζουν τον κύκλο εργασίας του (\textenglish{duty cycle}),
δηλαδή το τμήμα της περιόδου κατά το οποίο το παραγόμενο σήμα βρίσκεται σε
λογικό 1.

Η παραγωγή PWM υποστηρίζεται από τρεις ρυθμίσεις λειτουργίας των
Χρονομετρητών\slash Απαριθμητών, τις \textenglish{Fast PWM mode, Phase Correct
PWM mode (PCPWM)} και \textenglish{Phase and Frequency Correct PWM mode
(PFCPWM)}, εκ των οποίων οι δύο τελευταίες ενδείκνυνται για τον έλεγχο κινητήρων
εν γένει, και αυτό επειδή το παραγόμενο σήμα ανταποκρίνεται πιο ομαλά στις
αλλαγές της τιμής TOP \parencite[126,128]{atmel13}. Οι PCPWM και PFCPWM, εφόσον
η τιμή TOP -- η τιμή μέχρι την οποία αυξάνει ο TCNTn πριν αρχίσει τη φθίνουσα
πορεία του -- διατηρείται σταθερή κατά τη διάρκεια λειτουργίας του μετρητή,
είναι πανομοιότυπες \parencite[127]{atmel13}. Στην περίπτωση της υλοποίησης, η
συχνότητα του παραγόμενου σήματος είναι σταθερή, όπως έχει αναφερθεί, στα 50Hz
και, συνεπώς, το ίδιο ισχύει για την τιμή TOP. Ως αποτέλεσμα, οι λειτουργίες
PCPWM και PFCPWM είναι ισοδύναμες για τις ανάγκες της υλοποίησης και μπορεί να
προτιμηθεί είτε η μία είτε η άλλη, στην περίπτωση που επιλεγεί ο Timer/Counter1,
ή η PCPWM, στην περίπτωση που επιλεγεί ο Timer/Counter0, καθώς σύμφωνα με τις
επιλογές παραγωγής κυματομορφής της \textcite[107]{atmel13}, ο Timer/Counter0
υποστηρίζει μόνο αυτήν.
