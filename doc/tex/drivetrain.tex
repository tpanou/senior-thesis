
\section{Υποσύστημα κίνησης}


%
% Κινητήρες
%

\subsection{Παραγωγή σήματος\slash κίνησης}

Σύμφωνα με εγχειρίδιο χρήσης της \textcite{hitec02}, όλοι οι κινητήρες της
λειτουργούν σε τάση 4.8--6V δεχόμενοι τετραγωνικό σήμα ελέγχου με διαμόρφωση
PWM (\textenglish{Pulse-Width Modulation}) 3--5V σε συχνότητα ανανέωσης 50Hz.
Η διάρκεια των παλμών κυμαίνεται μεταξύ 0.9ms και 2.1ms, με παλμούς των 1.5ms να
χρησιμοποιούνται για την ακινητοποίησή τους \parencite{hitec02}.

Οι παραγωγή των παλμών μπορεί να πραγματοποιηθεί είτε με λογισμικό είτε από
κάποιο διαθέσιμο κύκλωμα Χρονομετρητή\slash Απαριθμητή (\textenglish{Timer\slash
Counter}) του μικροελεγκτή. Το κύριο πλεονέκτημα του λογισμικού είναι η
ελευθερία επιλογής οποιουδήποτε ακροδέκτη για την έξοδο του σήματος καθώς και η
δυνατότητα παραγωγής πολλαπλών τέτοιων σημάτων που περιορίζονται από το συνολικό
πλήθος ακροδεκτών ή τη συχνότητα του ρολογιού συστήματος (όποιο είναι
μικρότερο).

Αντιθέτως, στην περίπτωση χρήσης κυκλωμάτων, το παραγόμενο σήμα διοχετεύεται
στους συγκεκριμένους ακροδέκτες με τους οποίους είναι το καθένα εσωτερικά
συνδεδεμένο και, σαφώς, το μέγιστο πλήθος παράλληλων σημάτων περιορίζεται από το
συνολικό αριθμό τέτοιων κυκλωμάτων. Επίσης, κάθε Χρονομετρητής\slash Απαριθμητής
υπόκειται σε πρόσθετους περιορισμούς, όπως το εύρος τιμών που υποστηρίζει ο
καθένας (για παράδειγμα, ανάλυση 8- ή 16-bit).

Ωστόσο, στη χρήση κυκλώματος Χρονομετρητή\slash Απαριθμητή συγκαταλέγονται και
ορισμένα σημαντικά πλεονεκτήματα.
Καθότι το σήμα παράγεται από ξεχωριστό κύκλωμα, η CPU του μικροελεγκτή είναι
διαθέσιμη για την εκτέλεση οποιασδήποτε άλλης εργασίας (για παράδειγμα, κάποιας
ρουτίνας εξυπηρέτησης διακοπής) παράλληλα με την παραγωγή του σήματος, χωρίς να
απαιτείται πολύπλεξη εργασιών (ώστε να αποδίδονται κβάντα τόσο στη γεννήτρια
σήματος όσο και σε κάποια άλλη εργασία). Ένα δεύτερο, λιγότερο σημαντικό για την
προκειμένη υλοποίηση, πλεονέκτημα είναι η δυνατότητα επίτευξης υψηλότερων
συχνοτήτων με πολύ λιγότερο θόρυβο (\textenglish{jitter}) και μεγαλύτερη
ακρίβεια.

Τελικά, για την παραγωγή σήματος PWM των κινητήρων επιλέγεται η χρήση κυκλώματος
Χρονομετρητή\slash Απαριθμητή καθώς, επιπροσθέτως των ανωτέρω πλεονεκτημάτων,
παρουσιάζει υψηλότερο ενδιαφέρον, από εκπαιδευτικής πλευράς.
