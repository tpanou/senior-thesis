\section{Ημερολόγιο μετρήσεων}

\subsection{Ομάδα εγγραφών}
Τυπικά, δοθέντων δύο ημερομηνιών, το Ημερολόγιο είναι σε θέση να επιστρέψει όλες
τις ενδιάμεσες εγγραφές. Ωστόσο, η άμεση και μαζική επιστροφή τους είναι
απαγορευτική λόγω των περιορισμένων πόρων του μικροελεγκτή καθώς απαιτείται
δέσμευση χώρου στην κύρια μνήμη, δυνητικά, για όλες τις εγγραφές του
Ημερολογίου.
%
%Επιπλέον, απαιτείται
%η προσπέλαση του υποκείμενου αποθηκευτικού μέσου για την ανάκτηση εγγραφών εκ
%των οποίων μόλις ένα μικρό μέρος μπορεί να έχει ενδιαφέρον στην εκάστοτε
%περίπτωση. Για παράδειγμα, παρότι έχουν δοθεί δύο ημερομηνίες που καλύπτουν όλο
%το εύρος των εγγραφών, μπορεί να πρέπει να επιστραφούν σε μία εξωτερική οντότητα
%μόνο οι πρώτες δέκα, καθώς τα αποτελέσματα έχουν ζητηθεί να επιστραφούν
%σελιδοποιημένα. Είναι σαφές πως μία τέτοια τακτική μειώνει σημαντικά την απόδοση
%του συστήματος.

Αντιθέτως, η προσπέλαση των εγγραφών πραγματοποιείται σε βήματα. Αρχικά,
εκτελείται αναζήτηση βάσει των επιθυμητών ημερομηνιών από όπου επιστρέφεται το
πλήθος των εγγραφών που εντοπίστηκαν, καθώς και μία δομή \verb~LogRecordSet~. H
δομή αυτή αναπαριστά το σύνολο (ή ομάδα) των εντοπισμένων εγγραφών.
Ωστόσο, στην πραγματικότητα, περιέχει μόνο την πληροφορία που προσδιορίζει τη
θέση στο αποθηκευτικό μέσο όπου βρίσκεται η επόμενη εγγραφή του ιδεατού αυτού
συνόλου.
Για την οποιαδήποτε επενέργεια στις πραγματικές εγγραφές, παρέχονται ξεχωριστές
συναρτήσεις, κάθε μία δεχόμενη τη δομή \verb~LogRecordSet~. Με τον τρόπο αυτό
γίνεται γνωστή η κατάσταση του τρέχοντος συνόλου ώστε να προσαρμόζεται
καταλλήλως η συμπεριφορά της εκάστοτε συνάρτησης, ενώ δίνεται η δυνατότητα
ενημέρωσης της δομής μετά την ολοκλήρωση των εργασιών.

\subsection{Οργάνωση εγγραφών}

Στο σημείο αυτό αναλύεται πώς διαχειρίζεται το Ημερολόγιο τον αφιερωμένο
αποθηκευτικό χώρο για τις εγγραφές του. Παρότι περιγράφονται δύο σχεδιασμοί,
ωστόσο, αποτελούν όψεις του ίδιου νομίσματος· και οι δύο υλοποιούνται σε κώδικα
και λειτουργούν ο ένας πάνω από τον άλλο, προσδίδοντας διαφορετικά επίπεδα
αφαίρεσης. Στο πρώτο κομμάτι περιγράφεται  ενώ στο δεύτερο, πώς υλοποιείται η
διαχείριση των εγγραφών σε φυσικό επίπεδο.

\subsubsection{Επίπεδο κυλιόμενου πίνακα}
Σε υψηλό αφαιρετικό επίπεδο, ο αποθηκευτικός χώρος νοείται ως ένας μονοδιάστατος
πίνακας, εκτεταμένος σε διαδοχικές θέσεις μνήμης, κάθε στοιχείο του οποίου είναι
μία εγγραφή. Οι εγγραφές τοποθετούνται σε αύξουσα σειρά βάσει της ημερομηνίας
τους, με την παλαιότερη να βρίσκεται πάντα στην πρώτη θέση του πίνακα.

Υπό φυσιολογικές συνθήκες, κάθε νέα εγγραφή διαθέτει ημερομηνία μεγαλύτερη των
υπαρχόντων και, συνεπώς, προστίθεται μετά την τρέχουσα τελευταία εγγραφή.
Ωστόσο, η τροποποίηση των ρυθμίσεων της συσκευής -- για την ακρίβεια της
ημερομηνίας\slash ώρας --  είναι πιθανό να προκαλέσει την παραγωγή νέων εγγραφών
με ημερομηνία που προηγείται ορισμένων αποθηκευμένων εγγραφών. Η νέα εγγραφή
διατηρείται πάντα, σύμφωνα με την πεποίθηση ότι οι νέες ρυθμίσεις που προκαλούν
αυτήν τη χρονική ασυνέχεια, έχουν γίνει με σκοπό τη διόρθωση μίας εσφαλμένης
κατάστασης της συσκευής. Το ερώτημα τίθεται για τις παλαιότερα αποθηκευμένες
εγγραφές που διαθέτουν, πλέον, νεότερη ημερομηνία σε σχέση με κάποια νέα
εγγραφή.

Εάν διατηρούνται, τότε μετρήσεις που έχουν γίνει σε παρελθοντικό χρόνο
αναμιγνύονται με τις νέες μετρήσεις, ενδεχομένως, νοθεύοντας τα αποτελέσματά
τους, εφόσον υπάρχουν αποκλίσεις μεταξύ παλαιών και νέων. Για την αποφυγή
τέτοιων αβεβαιοτήτων, το Ημερολόγιο απορρίπτει τις προϋπάρχουσες εγγραφές, η
ημερομηνία των οποίων έπεται κάποιας νέας, προς αποθήκευση, εγγραφής.

Οι επιλογές που έχουν περιγραφεί μέχρι στιγμής, καθιστούν δυνατή την προσπέλαση
υπαρχόντων και την εισαγωγή νέων εγγραφών σε σταθερό χρόνο, ενώ επιτρέπουν την
εύρεση εγγραφών βάσει ημερομηνίας σε χρόνο $O(\log n)$ (μέσω δυαδικής
αναζήτησης). Και τα δύο είναι χαρακτηριστικά που εγγυούνται μειωμένο πλήθος
προσβάσεων στη συσκευή μόνιμης αποθήκευσης, η οποία, κατά πάσα πιθανότητα,
χαρακτηρίζεται από υψηλότερους χρόνους προσπέλασης. Ένα δεύτερο πλεονέκτημα
είναι η κατά το μέγιστο αξιοποίηση του αφιερωμένου αποθηκευτικού χώρου καθώς
αποθηκεύονται μόνο πραγματικά δεδομένα και όχι δευτερεύοντα βοηθητικά (όπως, για
παράδειγμα, στην περίπτωση λίστας). Βέβαια, όλα αυτά είναι άμεση απόρροια της
έλλειψης ανάγκης για ενημέρωση και διαγραφή εγγραφών.

Ένα τελευταίο χαρακτηριστικό του Ημερολογίου είναι η αντιμετώπιση της εξάντλησης
του διαθέσιμου αποθηκευτικού χώρου. Σε αυτήν την περίπτωση, επιλέγεται η
απόρριψη της παλαιότερης εγγραφής με την (νοητή) μετακίνηση του πίνακα κατά μία
θέση ώστε να δημιουργηθεί μία νέα θέση στο τέλος του (εξού και ο χαρακτηρισμός
ως «κυλιόμενος»). Σαφώς, ο αντίστοιχος μηχανισμός αναλαμβάνεται από χαμηλότερο
επίπεδο της υλοποίησης. Ωστόσο, αξίζει να σημειωθεί ότι στο μεγαλύτερο κομμάτι
της υλοποίησης ο αποθηκευτικός χώρος αντιμετωπίζεται με αυτόν τον τρόπο· ως ένας
πίνακας όπου το πρώτο του στοιχείο είναι η παλαιότερη εγγραφή, το πραγματικό
περιεχόμενο της οποίας αλλάζει ανά πάσα στιγμή, και ότι οι εγγραφές εκτείνονται
σε αύξουσα σειρά και καλύπτουν μερικώς ή πλήρως τις θέσεις του πίνακα.

\subsubsection{Επίπεδο κυκλικής ουράς}

Επειδή κύριας σημασίας είναι η διατήρηση των τελευταίων μετρήσεων, σε περίπτωση
εξάντλησης του διαθέσιμου αποθηκευτικού χώρου, οι νέες εγγραφές αντικαθιστούν
τις παλαιότερες. Η συμπεριφορά αυτή μπορεί να αποδοθεί από μία απλουστευμένη
μορφή δακτυλίου (ή κυκλικής ουράς), όπου επιτρέπεται μόνο η εισαγωγή στοιχείων
στη δομή με υποστήριξη επικάλυψης των παλαιοτέρων. Βέβαια στην πραγματικότητα, ο
χώρος αποθήκευσης είναι ένας πίνακας από διαδοχικά Byte ή, λίγο πιο αφαιρετικά,
από διαδοχικές θέσεις μεγέθους \verb~LogRecord~. Προκειμένου να υλοποιηθεί η
συμπεριφορά του δακτυλίου, απαιτείται η τήρηση ενός δείκτη που προσδιορίζει τη
θέση του πρώτου στοιχείου καθώς και ενός δείκτη για την τελευταία. Ωστόσο, αντί
για δείκτη τέλους, τηρείται το πλήθος των διαθέσιμων εγγραφών.
