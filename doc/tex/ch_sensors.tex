\chapter{Αισθητήρες}

\section{Οπτικοί κωδικοποιητές}

Σύμφωνα με έκδοσης της \textcite[12]{drc76}, οπτικοί κωδικοποιητές περιστροφικής
κίνησης, παραδοσιακά, κατασκευάζονται με την προσάρτηση ενός περιφερειακά
διάτρητου δίσκου στον άξονα κίνησης εκατέρωθεν του οποίου διατάσσεται αντικριστό
ζεύγος πομπού και δέκτη υπέρυθρων ακτίνων. Καθώς ο δίσκος περιστρέφεται ως
αποτέλεσμα κίνησης του άξονα, η ύπαρξη ή έλλειψη οπής επαναφέρει ή αποκόπτει την
επικοινωνία μεταξύ πομπού-δέκτη προκαλώντας εναλλαγές στην έξοδο του δέκτη
\parencite[12]{drc76}.

Στην απλούστερη υλοποίηση, ο αισθητήρας αναγνωρίζει τη μετάβαση από τη μία θέση
στην επόμενη ενώ είναι αδύνατο να αναχθεί από το σήμα και μόνον, είτε η φορά
περιστροφής είτε η τρέχουσα γωνιακή μετατόπιση του άξονα· ο ελεγκτής είναι
υπεύθυνος για την εξαγωγή αυτών των συμπερασμάτων \parencites[5--6]{lynch02}
[13]{drc76}. Στην περίπτωση αυτή, ο κωδικοποιητής αποκαλείται
\emph{προσαυξητικός}\index{προσαυξητικός κωδικοποιητής} (\emph{incremental})
\parencite[5]{lynch02}.

Ωστόσο, είναι δυνατό να κατασκευαστεί \emph{απόλυτος} (\emph{absolute})
κωδικωποιητής\index{κωδικοποιητής απόλυτης μετατόπισης} μετατόπισης, κάνοντας
χρήση πολλαπλών ζευγών πομπού-δέκτη και ενός δίσκου υποδιαιρεμένου σε διακριτές
θέσεις που αποτελούνται από μοναδικό συνδυασμό οπών \parencites[6]{lynch02}. Ο
κάθε αισθητήρας παράγει έξοδο ανεξάρτητη από τους υπολοίπους βάσει των οπών που
του αντιστοιχούν, ενώ η συνδυαστική έξοδος όλων των αισθητήρων περιγράφει τον
τρέχοντα συνδυασμό οπών και συνεπώς τη γωνιακή μετατόπιση του δίσκου
\parencites[6]{lynch02}.

Για τη σύνθεση ενός κωδικοποιητή που κάνει χρήση οπτικών αισθητήρων,
χρησιμοποιούνται ζεύγη πομπού και δέκτη υπέρυθρων ακτίνων. Το κάθε ζεύγος μπορεί
να αποτελείται από ανεξάρτητα, μεταξύ τους, στοιχεία ή να βρίσκονται
ενσωματωμένα σε ειδική θήκη που διευκολύνει την τοποθέτησή τους.

Υπάρχουν διατάξεις που τοποθετούν αντικριστά το ζεύγος πομπού και δέκτη
σχηματίζοντας έναν κενό χώρο μεταξύ τους στον οποίο μπορεί να εισέρχεται
εξωτερικό αντικείμενο, διακόπτοντας την επικοινωνία τους. Τέτοιοι αισθητήρες
αναφέρονται ως \emph{φωτοδιακόπτες}\index{φωτοδιακόπτης}
(\emph{photointerrupter}) \parencite[3]{lynch02} και αποτελούν τη διάταξη που
έχει παρουσιαστεί στα μέχρι τώρα παραδείγματα.

Σε εναλλακτική διάταξη, πομπός και δέκτης είναι μεταξύ τους παρακείμενοι με την
επικοινωνία τους να είναι δυνατή μόνο εφόσον οι εκπεμπόμενες ακτίνες ανακλαστούν
σε εξωτερική επιφάνεια. Τέτοιοι αισθητήρες αναφέρονται ως \emph{ανακλαστικοί}
\index{ανακλαστικός αισθητήρας} (\emph{reflective}) \parencite[3]{lynch02}.
Η έξοδος του δέκτη επηρεάζεται άμεσα από την ένταση των προσπίπτουσων ακτίνων η
οποία, με τη σειρά της, εξαρτάται από τις ανακλαστικές ιδιότητες και την
απόσταση της εξωτερικής επιφάνειας \parencite{vishay06}. Για την κατασκευή
οπτικού κωδικοποιητή κάνοντας χρήση αισθήτηρα τέτοιας διάταξης, ο προσαρτημένος
στον άξονα περιστροφής δίσκος είναι χωρισμένος σε τμήματα διαφορετικού και
εναλλασσόμενου συντελεστή ανάκλασης ώστε με την περιστροφή του να επηρεάζεται η
ένταση των προσπίπτουσων ακτίνων στον κάθετο ως προς το δίσκο αισθητήρα, και,
συνεπώς, η έξοδός του \parencite[11]{vishay02}.


